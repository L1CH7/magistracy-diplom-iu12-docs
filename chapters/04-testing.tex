\chapter{Экспериментальное исследование системы}

В данной главе приводятся результаты экспериментального исследования разработанной системы. Целью испытаний является подтверждение соответствия функциональных и нефункциональных требований, сформулированных в техническом задании, а также оценка предельных характеристик масштабируемости и надежности.

Исследование проводилось на реальном графе дорожной сети Московской агломерации. Особое внимание уделено анализу компромиссов между производительностью (Latency) и гибкостью маршрутизации в условиях динамически изменяющейся дорожной обстановки.

\section{Методология экспериментального исследования}

\subsection{Характеристика тестового полигона}

В качестве полигона для испытаний был использован граф дорожной сети Московской агломерации, включающий территорию МКАД и пригородную зону. Выбор данной области обусловлен высокой плотностью графа и наличием сложных топологических структур (многоуровневые развязки, кольцевые магистрали, мосты), создающих максимальную вычислительную нагрузку на алгоритмы обхода.

Параметры тестовой области:
\begin{itemize}
    \item \textbf{долгота} от $37.30^{\circ}$E до $37.90^{\circ}$E;
    \item \textbf{широта} от $55.45^{\circ}$N до $56.00^{\circ}$N.
\end{itemize}

\subsection{Аппаратная конфигурация тестового стенда}

Характеристики аппаратного обеспечения, использованного при проведении экспериментов:
\begin{itemize}
    \item центральный процессор AMD Ryzen 5 5600H (архитектура Zen 3, 6 физических ядер, 12 потоков, базовая частота 3,3~ГГц);
    \item оперативная память 16~ГБ DDR4-3200 (двухканальный режим), из которых 4~ГБ выделено под буферный пул PostgreSQL (\texttt{shared\_buffers});
    \item дисковая подсистема на базе NVMe SSD (M.2, PCIe 3.0 x4) со скоростью последовательного чтения до 3500~МБ/с;
    \item графический ускоритель NVIDIA GeForce RTX 3060 Laptop (6~ГБ GDDR6), задействованный для рендеринга векторных тайлов;
    \item операционная система Linux Arch (ядро 6.17.5-arch1-1) с файловой системой \texttt{ext4}.
\end{itemize}

\subsection{Инструментарий нагрузочного тестирования}

Для проведения экспериментов был разработан специализированный программный комплекс на языке Python, архитектура которого обеспечивает минимизацию накладных расходов на генерацию нагрузки. Программный комплекс, разработанный для проведения экспериментов, включает ряд функциональных компонентов.
\begin{itemize}
    \item Модуль оркестрации тестирования~-- фреймворк \texttt{pytest} с модулем \texttt{pytest-xdist} для параллельного запуска тестов. Использовалось 12 независимых воркеров (процессов), что соответствует числу логических ядер процессора, для создания максимальной конкурентной нагрузки.
    \item Сетевой транспортный модуль~-- асинхронный HTTP-клиент \texttt{httpx}, настроенный в режиме пулинга соединений. Повторное использование TCP-соединений исключает накладные расходы на Handshake и TLS Negotiation при каждом запросе, позволяя измерить <<чистое>> время обработки запроса на сервере.
    \item Сбор и анализ метрик~-- агрегация результатов (Latency, Throughput, Error Rate) производится в формате CSV с последующей статистической обработкой в библиотеке \texttt{pandas}.
\end{itemize}

\subsection{Стратегии выборки данных}

Для исключения эффекта <<горячего кэша>> и обеспечения статистической репрезентативности результатов, в модуле \texttt{utils.py} были реализованы две различные стратегии генерации пар точек старта и финиша.

Для нагрузочных тестов применяется равномерная сеточная выборка. Координаты точек привязываются к узлам регулярной сетки с шагом $\approx 0.05^{\circ}$ (около 5 км) с использованием функции \texttt{ST\_SnapToGrid}. Данный подход гарантирует равномерное покрытие всей площади исследуемого полигона и предотвращает локализацию запросов в одном районе, что могло бы искусственно завысить показатели производительности за счет кэширования страниц БД.

Для метрической оценки качества маршрутизации (коэффициенты извилистости и Жаккарда) используется стратегия удаленных точек. Генератор выбирает случайные пары точек с условием, что евклидово расстояние между ними превышает пороговое значение:
\begin{equation}
L_{euclid} > 4000 \text{ м}.
\end{equation}

Фильтрация коротких маршрутов необходима для исключения тривиальных внутриквартальных путей, где альтернативные варианты проезда физически отсутствуют. Это позволяет сфокусировать тестирование на магистральных направлениях, где эффективность алгоритма маршрутизации критически важна.

\section{Анализ производительности и масштабируемости}

Ключевым критерием эффективности навигационной системы является способность обрабатывать запросы пользователей с приемлемой задержкой (Soft Real-Time) даже в условиях пиковых нагрузок.

\subsection{Влияние «холодного старта» и дискового ввода-вывода}

В ходе Latency Test был выявлен значительный разброс времени отклика в зависимости от состояния кэша базы данных.
При первом запуске («холодный старт») время построения маршрута через всю Москву (МКАД Север $\to$ МКАД Юг, $\approx 45$ км) достигало \textbf{40 секунд}. Профилирование показало, что 92\% времени тратится на подъем данных с диска (NVMe SSD) в оперативную память. Алгоритм \texttt{pgr\_dijkstra} требует доступа к миллионам строк таблицы ребер, и даже быстрый SSD становится узким местом (IO Wait).

Однако при повторном запросе («прогретый кэш») время сокращалось до \textbf{2.5 секунд}. Это подтверждает эффективность механизма Shared Buffers в PostgreSQL\cite{osti_distributed_caching_2018}. В условиях высокой нагрузки мультиагентной системы это обеспечивает необходимую надежность\cite{almutairi2025reliable}.
\begin{figure}[H]
    \centering
    \includegraphics[width=0.9\linewidth]{assets/images/latency_regression}
    \caption{Зависимость времени поиска от физического расстояния маршрута (log-log scale). Точки — отдельные запросы, линия — регрессионная модель.}
    \label{fig:latency-regression}
\end{figure}

Диаграмма рассеяния (рисунок~\ref{fig:latency-regression}) демонстрирует логарифмическую зависимость времени поиска от расстояния. Это достигается благодаря применению динамического ограничивающего прямоугольника (Dynamic BBOX).
Для коротких маршрутов (< 5 км) BBOX отсекает 99\% графа, и поиск занимает менее 100 мс. Для длинных маршрутов область поиска расширяется, но алгоритм все равно не сканирует весь граф целиком.

\subsection{Парадокс MVCC и вертикальная масштабируемость}

При проведении нагрузочного тестирования (Throughput Test) был зафиксирован неочевидный результат, названный «Парадоксом MVCC».
При увеличении количества параллельных клиентов с 1 до 16 (в соответствии с числом логических ядер CPU), среднее время отклика выросло всего на \textbf{9.7\%} (с 2.52 с до 2.76 с).
При этом утилизация процессора (CPU Load Average) выросла с 12\% до 87\%.

\begin{figure}[H]
    \centering
    \includemermaid[width=\linewidth, height=0.5\textheight, keepaspectratio]{04-mvcc-seq}
    \caption{Взаимодействие клиентов и БД в режиме конкурентности (MVCC)}
    \label{fig:mvcc-seq}
\end{figure}

Этот феномен объясняется архитектурой PostgreSQL:
\begin{itemize}
    \item Каждый запрос обслуживается отдельным процессом (Process-based architecture).
    \item Механизм MVCC (Multi-Version Concurrency Control) позволяет читающим транзакциям не блокировать друг друга (Readers never block readers).
    \item Асинхронный драйвер \texttt{asyncpg} эффективно мультиплексирует соединения, не создавая накладных расходов на переключение контекста в Python.
\end{itemize}

Таким образом, система демонстрирует практически линейную вертикальную масштабируемость вплоть до исчерпания ресурсов CPU.

\subsection{Анализ эффективности форматов передачи данных}

Сравнение объемов передаваемого трафика подтвердило необходимость использования бинарных форматов.
Текстовое представление геометрии (WKT, GeoJSON) приводит к раздуванию (Bloat) полезной нагрузки в 3-4 раза по сравнению с бинарными аналогами (WKB, MVT).

\begin{figure}[H]
    \centering
    \includegraphics[width=0.9\linewidth]{assets/images/throughput_complexity_3d}
    \caption{3D-поверхность зависимости пропускной способности от размера запроса и сложности графа}
    \label{fig:throughput-complexity}
\end{figure}

Анализ поверхности производительности (рисунок~\ref{fig:throughput-complexity}) показывает, что переход на WKB (Binary) снижает накладные расходы на сериализацию на 70\% по сравнению с WKT, что особенно заметно на сложных маршрутах (>1000 вершин).

Для мобильных клиентов, работающих в сетях с нестабильным соединением (3G/LTE), экономия 70\% трафика является критическим преимуществом, напрямую влияющим на воспринимаемую скорость работы приложения (User Experience).

\section{Оценка качества маршрутизации и надежности} \label{sec:quality}

Помимо скоростных характеристик, критически важным аспектом является воспринимаемое качество маршрутов -- их логичность, безопасность и предсказуемость.
Для количественной оценки качества использовались интегральные метрики извилистости и семантического разнообразия.

\subsection{Коэффициент извилистости}

Метрика извилистости $T$ определяется как отношение длины найденного маршрута к евклидову расстоянию между точками старта и финиша:
\begin{equation}
T = \frac{L_{route}}{L_{euclid}},
\end{equation}
\where{
    $L_{route}$ & длина найденного маршрута; \\
    $L_{euclid}$ & евклидово расстояние по прямой.
}

Гистограмма распределения значений коэффициента извилистости представлена на рисунке \ref{fig:tortuosity-hist}.

\begin{figure}[H]
    \centering
    \includegraphics[width=0.8\linewidth]{assets/images/latency_tortuosity_hist}
    \caption{Гистограмма коэффициента извилистости (Tortuosity) для найденных маршрутов}
    \label{fig:tortuosity-hist}
\end{figure}

Анализ гистограммы показал, что медианное значение составляет \textbf{$T \approx 1.36$}. Распределение имеет логнормальное характер, что позволяет выделить две группы маршрутов.

\begin{enumerate}
    \item Основной пик ($T \in [1.2, 1.7]$) соответствует эффективному движению по уличной сетке регулярной планировки. Близость к единице подтверждает, что механизм адаптивного окна корректно ограничивает область поиска, не позволяя алгоритму Dijkstra уходить в глубокий перебор.
    \item Длинный хвост ($T > 2.5$) не является аномалией, а отражает корректную работу алгоритма в условиях естественных преград, например, изгибы Москвы-реки, железнодорожные пути. В таких ситуациях прямой путь геометрически невозможен, поэтому система строит оптимальный топологический объезд.
\end{enumerate}

\subsection{Семантическое разнообразие маршрутов}

Для оценки эффективности метода итеративных штрафов при поиске альтернативных путей использовалась выборка случайной пары точек на расстоянии более $4$ км. 

Метрикой оценки был выбран индекс Жаккарда, который вычисляется по формуле:

\begin{equation}
J(R_1, R_2) = \frac{|E(R_1) \cap E(R_2)|}{|E(R_1) \cup E(R_2)|},
\end{equation}
\where{
    $E(R_1)$ & множество ребер основного маршрута; \\
    $E(R_2)$ & множество ребер альтернативного маршрута.
}
Дополнительно для оценки топологической независимости альтернатив использовался индекс разнообразия путей (Path Diversity Index), рассчитываемый по формуле:

\begin{equation}
PDI = \frac{\left| \bigcup_{i=1}^{K} E(R_i) \right|}{\sum_{i=1}^{K} \left| E(R_i) \right|},
\end{equation}
\where{
    $E(R_i)$ & множество ребер $i$-го маршрута; \\
    $K$ & количество найденных альтернативных маршрутов.
}

Значение $PDI \to 1$ указывает на то, что маршруты практически не имеют общих участков (максимальное разнообразие), тогда как $PDI \to 1/K$ свидетельствует о полном совпадении путей.

Для визуализации результатов эксперимента были построены два графика. На рисунке \ref{fig:jaccard-vs-dist} показана зависимость коэффициента Жаккарда от расстояния между точками, а на рисунке \ref{fig:pdi-dist} -- распределение индекса разнообразия путей.

\begin{figure}[H]
    \centering
    \includegraphics[width=0.8\linewidth]{assets/images/jaccard_vs_distance}
    \caption{Зависимость коэффициента Жаккарда от расстояния между точками}
    \label{fig:jaccard-vs-dist}
\end{figure}

\begin{figure}[H]
    \centering
    \includegraphics[width=0.8\linewidth]{assets/images/pdi_distribution}
    \caption{Распределение индекса разнообразия путей}
    \label{fig:pdi-dist}
\end{figure}

Интерпретация полученных данных позволяет сделать следующие выводы:

\begin{itemize}
    \item средний коэффициент Жаккарда $J \approx 0.0278$ (менее 3~\%).
    \item индекс разнообразия путей (PDI) более $0.93$.
\end{itemize}

Низкий процент перекрытия доказывает, что алгоритм находит топологически независимые магистрали (например, предлагая выбор между Ленинградским и Дмитровским шоссе), а не локальные вариации одного и того же пути. Локальные всплески сходства на графике объясняются наличием <<бутылочных горлышек>> (мостов и тоннелей), где физически отсутствуют альтернативные маршруты.

\subsection{Коэффициент успешности построения}

Визуализация зависимости вероятности успешного построения маршрута от его длины и количества агентов приведена на рисунке \ref{fig:yield-heatmap}. Как видно из графика, успешность построения маршрута практически не зависит от количества запрашиваемых альтернатив $K$.

\begin{figure}[H]
    \centering
    \includegraphics[width=0.9\linewidth]{assets/images/throughput_yield_heatmap}
    \caption{Тепловая карта успешности построения }
    \label{fig:yield-heatmap}
\end{figure}

Статистический анализ успешности построения маршрутов позволил выявить ряд закономерностей:

\begin{itemize}
    \item показатель успешности при $N=2$ составляет \textbf{95~\%};
    \item при маршруте с $N=5$ снижается до \textbf{69.2~\%};
    \item при варьировании числа альтернатив ($K$) показатель нахождения маршрута остается стабильным.
\end{itemize}

В среднем, уровень отказов и частичных результатов составил 14.8~\%. Анализ причин показал, что основные факторы отказов топологические, а не алгоритмические.

\begin{enumerate}
    \item Граничные эффекты~-- тестовый полигон ограничен, потому маршруты на периферии часто оптимально проходят через область (за границами BBox), что приводит к невозможности построения пути внутри полигона.
    \item Компромисс точности~-- использован агрессивный коэффициент расширения динамического окна $\delta(d) = 0.3 \cdot d$. Согласно исследованию \cite{zhou2017bounding}, это позволяет значительно ускорить поиск, но может отсекать валидные маршруты, требующие глубокого объезда.
    \item Проблема связности~-- вероятность попадания одной из точек старта/финиша в изолированный подграф (на закрытую территорию) растет экспоненциально с увеличением числа агентов $N$. Проблема несвязных компонент графа характерна для реальных данных OSM и требует процедур топологической фильтрации (прунинга) \cite{opentripplanner2023pruning}. Данная операция заключается в выделении максимальной компоненты связности дорожной сети и исключении изолированных участков, попадание на которые делает невозможным построение маршрута.
\end{enumerate}

Таким образом, экспериментальное исследование подтвердило высокую надежность разработанного алгоритмического обеспечения. Общий показатель успешности построения маршрутов превышает 85~\%.
\section{Сравнительная характеристика решений}

Для позиционирования разработанной системы относительно существующих аналогов проведено прямое сравнение с эталонным движком с открытым исходным кодом OSRM. Сравнение характеристик для графа масштаба Москвы приведено в таблице \ref{tab:comparison-osrm}.

\begin{table}[H]
    \centering
    \caption{Сравнительный анализ архитектурных подходов}
    \label{tab:comparison-osrm}
    \small
    \begin{tabularx}{\linewidth}{|p{5cm}|X|X|}
        \hline
        \textbf{Характеристика} & \textbf{Разработанная система (PostgreSQL)} & \textbf{OSRM (C++)} \\ \hline
        Архитектура & На базе СУБД & Предрасчитанный граф \\ \hline        
        Задержка отклика (поиск) & $\approx 6$~с (Последовательно) / $\approx 1.2$~с (Параллельно) & $10-25$~мс \cite{neis2012routing} \\ \hline
        Потребление ОЗУ (Москва) & $370$~МБ & $500$~МБ \cite{osrm2018memory} \\ \hline
        Время обновления графа & В реальном времени & $5-15$~мин \cite{malagis2025osrm} \\ \hline
        Учет динамических весов & Нативная поддержка & Не поддерживается \\ \hline
    \end{tabularx}
\end{table}

Как видно из таблицы, OSRM выигрывает в скорости поиска маршрута на два порядка за счет использования предварительно рассчитанных индексов и хранения всего графа в оперативной памяти. Однако этот подход имеет критический недостаток -- статичность данных. Любое изменение весового коэффициента, например, перекрытие дороги из-за аварии, требует полной перестройки индекса, что занимает до 15 минут \cite{malagis2025osrm} на масштабах рассматриваемого графа Москвы.

Разработанная система, уступая в <<чистой>> скорости чтения, обеспечивает ключевое преимущество -- актуализацию топологии в режиме реального времени. Изменение весов выполняется одной транзакцией SQL \texttt{UPDATE} и мгновенно учитывается при построении следующего маршрута. 

Это делает предложенную архитектуру оптимальным выбором для систем логистики реального времени, где критически важна адаптивность к дорожной ситуации, а задержка отклика в пределах нескольких секунд является допустимой платой за актуальность данных.

\section{Выводы по главе}

Проведенное экспериментальное исследование подтвердило работоспособность системы на реальных данных. 
Архитектурное решение на базе PostgreSQL, несмотря на проигрыш в чистой скорости (Latency) специализированным in-memory решениям (OSRM), продемонстрировало высокую утилизацию ресурсов (CPU/RAM) и способность обрабатывать динамические изменения графа в реальном времени.
Выявленный "Парадокс MVCC" подтвердил правильность выбора асинхронной модели работы с базой данных, обеспечивающей линейную масштабируемость под нагрузкой.
