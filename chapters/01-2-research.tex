\section{Математическая модель транспортной сети}

В основе навигационной системы лежит представление дорожной сети в виде графа $G = (V, E)$. Целевая функция поиска маршрута $P^*$ заключается в минимизации суммарной стоимости прохождения ребер:
\begin{equation}
P^* = \arg \min_{P \in \mathcal{P}_{st}} \sum_{e \in P} w(e, t),
\end{equation}
где $w(e, t)$ — весовая функция, зависящая от времени. Она рассчитывается как отношение длины участка $L_e$ к эффективной скорости $v_{eff}$, которая моделируется на основе диаграммы \textbf{Гриншилдса} (Greenshields Traffic Flow Model)\cite{trb2015fundamental}:
\begin{equation}
    v = v_f \left( 1 - \frac{k}{k_j} \right),
\end{equation}
где $v_f$ — скорость свободного потока, а $k_j$ — плотность затора.

Физическая модель данных, реализующая граф в реляционной СУБД, представлена на рисунке \ref{fig:er-schema}.

\begin{figure}[H]
    \centering
    \include{assets/diagrams/03-graph-schema}
    \caption{ER-диаграмма схемы базы данных: хранение графа $G(V,E)$ и метаданных OSM}
    \label{fig:er-schema}
\end{figure}

\section{Методика формирования графа дорог}

Для обеспечения корректности маршрутизации необходимо фильтровать исходные данные OpenStreetMap. В систему загружаются только объекты, имеющие значение для автомобильной навигации. Процесс фильтрации реализуется на языке Overpass QL и включает 4 этапа.

\begin{enumerate}
    \item \textbf{Белый список дорог (Whitelist).} Выбираются только линии с тегом \texttt{highway}, соответствующие асфальтированным дорогам (motorway, trunk, primary) и жилым улицам (residential). Пешеходные тропы и сервисные проезды исключаются.
    \item \textbf{Ограничения поворотов (Turn Restrictions).} Извлекаются отношения (relations) типа \texttt{restriction}, определяющие запреты маневров (например, only\_right\_turn).
    \item \textbf{Точечные барьеры.} Учитываются узлы с тегом \texttt{barrier} (шлагбаумы, ворота), которые могут блокировать проезд.
    \item \textbf{Ограничения доступа (Access Restrictions).} Исключаются частные территории (\texttt{access=private}) и выезды с парковок (\texttt{service=driveway}), так как они часто создают ложные "срезы" маршрутов через дворы.
\end{enumerate}

Для реализации данной фильтрации применяется стратегия \textbf{рекурсивного извлечения данных} (Recursive Extraction Strategy), обусловленная особенностями объектной модели OSM. Процесс формирования выборки $D_{raw}$ описывается последовательностью операций:

\begin{enumerate}
    \item \textbf{Селекция метаданных (\texttt{out body}).} На первом этапе выполняется поиск идентификаторов объектов (Way, Node, Relation), удовлетворяющих атрибутивным фильтрам. Геометрия на данном этапе игнорируется для минимизации трафика.
    \item \textbf{Рекурсивное разрешение зависимостей (\texttt{>}).} Поскольку линии (Ways) в OSM состоят из ссылок на узлы (Nodes), а отношения (Relations) — из ссылок на линии и точки, выполняется операция топологического спуска (recurse down). Это гарантирует целостность графа.
    \item \textbf{Генерация скелетной геометрии (\texttt{out skel qt}).} Для передачи данных используется компактный формат <<скелета>>, содержащий только координаты узлов. Модификатор \texttt{qt} (Quasi-Tile) оптимизирует порядок сортировки данных для потоковой обработки.
\end{enumerate}

Таким образом, граф $G$ формируется из подмножества объектов OSM, удовлетворяющих условиям фильтрации:
\begin{equation}
    E \subset \{ w \in \text{OSM}_{ways} \mid \text{tag}(w) \in \text{Whitelist} \land \text{access}(w) = \text{public} \}
\end{equation}

Детальная реализация фильтра на языке запросов Overpass QL приведена в главе \ref{chap:implementation}.


\subsection{Метод пространственного секционирования (Grid Partitioning)}
Для решения проблемы исчерпания памяти при построении топологии графа применен алгоритм Grid Partitioning. Исходный граф разбивается на тайлы, что позволяет строить узловую сеть (\texttt{pgr\_nodeNetwork}) независимо для каждого фрагмента $P \times P$\cite{zhu2015gridgraph}.

\subsection{Динамическое ограничение области поиска}
Для оптимизации алгоритма Дейкстры используется метод \textbf{Goal Pruning}. Поиск выполняется только внутри ограничивающего прямоугольника (Bounding Box), рассчитываемого динамически:
\begin{equation}
\delta(d) = \max \left(0.015^\circ, d \cdot 0.3 \right).
\label{eq:dynamic-bbox}
\end{equation}
Это позволяет отсекать до 90\% нерелевантных ребер\cite{sturtevant2015bounding}.

\subsection{Поиск альтернативных маршрутов}
Применяется метод \textbf{Iterative Penalty}. После нахождения $k$-го маршрута веса его ребер умножаются на штрафной коэффициент $\mu$, вынуждая алгоритм искать топологически отличные пути\cite{akgun2013penalty}.

\subsection{Анализ связности графа}
Для исключения изолированных подграфов ("островов"), из которых невозможен выезд на основную сеть, используется алгоритм Тарьяна поиска сильно связных компонент (SCC)\cite{tarjan2020algorithm}.

\begin{figure}[H]
    \centering
    \includegraphics[width=0.95\linewidth]{components_map}
    \caption{Карта связности графа. Выделены изолированные кластеры (острова), исключенные из маршрутизации.}
    \label{fig:components-map}
\end{figure}
