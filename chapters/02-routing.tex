\section{Проблема построения графа: Out Of Memory}
\label{sec:oom-problem}

Стандартная функция \texttt{pgr\_nodeNetwork} из библиотеки \texttt{pgRouting} предназначена для автоматического построения топологии: она находит пересечения дорог и создает узлы графа. Алгоритм работает по принципу <<загрузить весь граф в память>>.

При переходе от тестового графа (район Хамовники, $\approx$ 5000 ребер) к полному графу Московской агломерации (более 250\,000 ребер) процесс сборки стабильно завершался с ошибкой \textbf{OOM (Out Of Memory)}.

\subsection{Анализ потребления ресурсов}

В таблице \ref{tab:oom-metrics} приведены метрики потребления ресурсов при использовании стандартного подхода.

\begin{table}[H]
    \centering
    \caption{Потребление ресурсов при сборке (Legacy подход)}
    \label{tab:oom-metrics}
    \begin{tabular}{|l|l|l|}
        \hline
        \textbf{Метрика} & \textbf{Значение} & \textbf{Источник} \\ \hline
        Потребление RAM & $> 12$ ГБ & \texttt{docker stats} (пик перед OOM) \\ \hline
        Время до падения & $\approx 2$ часа & Логи контейнера \\ \hline
        Доступная RAM & 16 ГБ & Аппаратное ограничение сервера \\ \hline
        Результат & Crash (Code 137) & Kernel OOM Killer \\ \hline
    \end{tabular}
\end{table}

Профилирование показало, что узким местом являются длинные магистрали (МКАД, ТТК, Ленинградское шоссе). В исходных данных OSM они представлены одной геометрией длиной 8--15 км.

\begin{code}{edge-stats}{SQL}{Запрос статистики длин ребер}
SELECT 
    PERCENTILE_CONT(0.5)  WITHIN GROUP(ORDER BY length_m) AS median,
    PERCENTILE_CONT(0.99) WITHIN GROUP(ORDER BY length_m) AS p99,
    MAX(length_m) AS max_len,
    COUNT(*) AS count
FROM raw_osm_edges;
\end{code}

Вариант 2: Простой код (текст в рамке)
% Аргументы: {label}{Подпись}
\begin{plaincode}{config-example}{Пример конфигурации}
server_host = 127.0.0.1
server_port = 8080
\end{plaincode}

Результат анализа длин ребер представлен в таблице \ref{tab:edge-len-stats}.

\begin{table}[H]
    \centering
    \caption{Статистика длин ребер ДО оптимизации}
    \label{tab:edge-len-stats}
    \begin{tabular}{|l|l|l|}
        \hline
        \textbf{Перцентиль} & \textbf{Длина (м)} & \textbf{Комментарий} \\ \hline
        Median (50\%) & 70.46 & Типичная улица \\ \hline
        P99 (99\%) & 841.80 & Начало <<хвоста>> распределения \\ \hline
        \textbf{Max} & \textbf{8373.13} & МКАД одним куском \\ \hline
    \end{tabular}
\end{table}

\section{Архитектура хранения данных}

Для обеспечения преемственности данных (Data Lineage) и высокой производительности поиска, база данных разделена на два логических контура: \textbf{OSM} (сырые данные) и \textbf{GRAPHS} (оптимизированная топология).

\begin{figure}[H]
    \centering
    % ER-диаграмма высокая, ограничиваем высоту
    \includemermaid[height=0.5\textheight]{02-er-schema}
    \caption{ER-диаграмма схемы данных (OSM vs GRAPHS)}
    \label{fig:er-schema}
\end{figure}

Ключевые детали реализации:
\begin{itemize}
    \item \textbf{[GiST] Indices:} Каждая таблица с геометрией снабжена пространственным индексом. Поиск пересечений без индекса на 250k ребер занимает более 10 часов, с индексом --- менее 15 минут.
    \item \textbf{Data Lineage:} Поле \texttt{osm\_way\_id} позволяет прокидывать любые теги из сырых данных в веса графа без перестроения топологии.
\end{itemize}

\section{Реальный процесс построения}

В отличие от стандартных решений, разработанный механизм построения графа прозрачен и оптимизирован под ограничения RAM (16 ГБ).

\begin{figure}[H]
    \centering
    % ТА САМАЯ ДИАГРАММА, КОТОРУЮ МЫ ЧИНИЛИ
    \includemermaid[height=0.3\textheight]{02-logic-flow}
    \caption{Алгоритм построения дорожного графа. Это очень длинный текст, чтобы проверить, что как он будет переноситься на следующую строку.}
    \label{fig:build-flow}
\end{figure}

Процесс включает следующие этапы:
\begin{enumerate}
    \item \textbf{Дискретизация:} Линии длиннее 500 м нарезаются на сегменты.
    \item \textbf{Разделение потоков:} Мосты и тоннели исключаются из процесса поиска пересечений.
    \item \textbf{Грид-Нодинг:} Мир разбивается на тайлы, пересечения ищутся изолированно в памяти.
    \item \textbf{Наследование атрибутов:} Восстановление названий улиц и скоростных режимов по пространственному индексу.
\end{enumerate}

\section{Решение: Grid Partitioning}

Вместо обработки всего графа целиком, карта разбивается на квадратные тайлы размером $0.05^\circ \times 0.05^\circ$ ($\approx 3\times 5$ км).

\begin{figure}[H]
    \centering
    \includemermaid[height=0.3\textheight]{02-grid-algo}
    \caption{Схема алгоритма Grid Partitioning}
    \label{fig:grid-algo}
\end{figure}

\subsection{Результаты оптимизации}

Сравнение производительности представлено в таблице \ref{tab:optim-results}.

\begin{table}[H]
    \centering
    \caption{Сравнение подходов к построению графа}
    \label{tab:optim-results}
    \begin{tabular}{|l|p{4cm}|p{4cm}|l|}
        \hline
        \textbf{Метрика} & \textbf{Legacy} & \textbf{Grid Partitioning} & \textbf{Улучшение} \\ \hline
        RAM & $> 12$ ГБ (OOM) & \textbf{370 МБ} & $\downarrow 32$x \\ \hline
        Время & $> 4$ часа & \textbf{13 минут} & $\downarrow 18$x \\ \hline
        Стабильность & Crash & Success & --- \\ \hline
    \end{tabular}
\end{table}

\section{Контроль качества графа}

\subsection{Поиск островов связности}
Частая ошибка --- создание изолированных подграфов. Для анализа связности использовалась функция \texttt{pgr\_connectedComponents}.

\begin{table}[H]
    \centering
    \caption{Анализ компонент связности}
    \label{tab:connectivity}
    \begin{tabular}{|l|l|l|}
        \hline
        \textbf{Компонента} & \textbf{Кол-во ребер} & \textbf{Доля} \\ \hline
        \textbf{Main (ID=1)} & 240,234 & \textbf{98.4\%} \\ \hline
        Island 2 & 1,823 & 0.7\% \\ \hline
        Island 3 & 892 & 0.4\% \\ \hline
    \end{tabular}
\end{table}

\section{Разделение потоков: Ground vs Elevated}

Для решения проблемы ложных пересечений (например, эстакада над улицей) применяется группировка по слоям.

\begin{table}[H]
    \centering
    \caption{Логика разделения слоев}
    \label{tab:layers}
    \begin{tabular}{|p{3cm}|p{6cm}|p{4cm}|}
        \hline
        \textbf{Группа} & \textbf{Критерий (OSM tags)} & \textbf{Обработка} \\ \hline
        \textbf{Group A (Ground)} & \texttt{layer=0} AND NOT bridge/tunnel & \texttt{pgr\_nodeNetwork} \\ \hline
        \textbf{Group B (Isolated)} & \texttt{bridge=yes} OR \texttt{tunnel=yes} & Только \texttt{ST\_Subdivide} \\ \hline
    \end{tabular}
\end{table}