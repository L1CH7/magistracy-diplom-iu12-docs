\chapter{Теоретические основы и проектирование системы динамической маршрутизации}
\label{chap:theory}

\section{Анализ задачи маршрутизации в динамической среде}

\subsection{Специфика маршрутизации в динамической среде}

Задача поиска оптимального маршрута в транспортной сети сводится к нахождению кратчайшего пути во взвешенном ориентированном графе:
\begin{equation}
    G = (V, E),
\end{equation}
\where{
    $V$ & множество вершин графа, соответствующих перекресткам и дорожным развязкам; \\
    $E$ & множество ориентированных ребер, представляющих дорожные сегменты; \\
    $w(e)$ & весовая функция ребра (стоимость проезда), которая в классической постановке считается константой.
}

Классический подход позволяет использовать эффективные алгоритмы предварительной обработки, такие как метод иерархий сокращений (Contraction Hierarchies) \cite{geisberger2008_researchgate}, сокращающий время запроса до миллисекунд.

Однако в условиях городской среды веса ребер $w(e, t)$ являются функцией времени, изменяясь под воздействием пробок, аварий или перекрытий дорог.

\subsection{Сравнительный анализ существующих решений}

Для выбора платформы был проведен сравнительный анализ существующих решений \cite{routexl2025comparison, gisops2018routing}. На современном рынке выделяются два основных класса систем.

\begin{enumerate}
    \item \textbf{Системы с предварительной обработкой в оперативной памяти} (OSRM, GraphHopper, Valhalla). Ориентированы на максимальную производительность поиска за счет сложного этапа предварительной подготовки данных. Граф компилируется в оптимизированные структуры данных, такие как иерархии сокращений или векторные тайлы.
    \item \textbf{СУБД-ориентированные решения} (pgRouting). Ориентированы на гибкость и работу с динамическими данными непосредственно в системе управления базами данных без предварительного построения индексов.
\end{enumerate}

В таблице \ref{tab:routing-engines-general} представлено сравнение архитектурных особенностей данных решений.

\begin{table}[H]
    \centering
    \caption{Сравнительный анализ архитектурных подходов к маршрутизации}
    \label{tab:routing-engines-general}
    \small
    \begin{tabularx}{\textwidth}{|p{0.18\linewidth}|X|X|X|}
        \hline
        \textbf{Решение} & \textbf{Алгоритм оптимизации} & \textbf{Механизм работы} & \textbf{Ограничения для динамических графов} \\ \hline
        \textbf{OSRM} & Метод иерархий сокращений (CH) / MLD & Строит иерархию сокращений во время предварительной обработки. & Любое изменение весов требует полной перестройки графа (до 1 часа для Москвы). \\ \hline
        \textbf{GraphHopper} & CH / Метод ориентиров (ALT) & Использует гибридный подход. Гибкий режим позволяет менять параметры, но без изменения топологии. & Высокое потребление оперативной памяти (Java Heap). Пересчет весов в реальном времени неэффективен. \\ \hline
        \textbf{Valhalla} & Иерархические тайлы & Граф разбит на фрагменты. \textbf{Динамический расчет стоимости} вычисляет веса на основе атрибутов фрагмента \cite{saki2022mapmatching}. & Обновление весов требует пересборки фрагментов и перезапуска сервиса. Максимальная частота обновления -- раз в неделю (исторические данные). \\ \hline
        \textbf{pgRouting} & \textbf{Отсутствует} (Алгоритм Дейкстры / A*) & Работает напрямую с топологией в таблицах PostGIS. Веса вычисляются SQL-запросом <<на лету>> \cite{crunchydata2026routing}. & Линейная зависимость времени поиска от размера графа. Требует высокой производительности дисковой подсистемы. \\ \hline
    \end{tabularx}
\end{table}

Как показывает анализ, решения, выполняющие вычисления в оперативной памяти (OSRM, GraphHopper, Valhalla), жертвуют возможностью мгновенного изменения весов ради производительности. В частности, система Valhalla, несмотря на заявленный механизм динамического расчета стоимости пути, для актуализации данных о трафике требует периодической пересборки графовых фрагментов и перезапуска сервиса маршрутизации: \textit{<<you would need to periodically update your tileset and restart your routing service to make the changes live>>} \cite{valhalla_issue4030}. Подобные архитектурные ограничения делают эти системы непригодными для задач мультиагентной навигации, где весовые коэффициенты ребер должны обновляться непрерывно по мере движения агентов.
% \textit{<<Вам потребуется периодически обновлять набор тайлов и перезапускать сервис маршрутизации для применения изменений>>} %

\section{Обоснование выбора СУБД-ориентированной архитектуры}

На основе анализа было принято решение использовать архитектуру, ориентированную на базы данных. Граф дорог хранится и обрабатывается непосредственно в реляционной СУБД PostgreSQL. Данный подход обладает рядом преимуществ для геоинформационных систем\cite{basargin2021pgrouting}:
\begin{enumerate}
    \item \textbf{Транзакционная целостность.} Маршруты строятся на основе согласованного снимка данных, обеспечиваемого механизмом многоверсионности (MVCC), даже если параллельно идут процессы обновления графа.
    \item \textbf{Единая точка истины.} Локационные и бизнес-данные (пробки, инциденты) находятся в одном хранилище, устраняя необходимость синхронизации с внешними поисковыми индексами\cite{percona2025postgis}.
\end{enumerate}

Расширение pgRouting, в свою очередь, обеспечивает атомарное изменение весов через стандартные SQL-операции (\texttt{UPDATE}), что критически важно для реализации динамической маршрутизации в условиях меняющегося трафика.

\section{Архитектура программного комплекса}

Система спроектирована по принципам сервис-ориентированной архитектуры, как показано на рисунке \ref{fig:system-arch}. Взаимодействие компонентов организовано следующим образом:
\begin{enumerate}
    \item \textbf{Шлюз доступа (Gateway)} принимает запросы клиентов и маршрутизирует их.
    \item \textbf{Сервис маршрутизации (Router)} выполняет SQL-запросы к базе данных для построения маршрутов.
    \item \textbf{Сервис обработки данных (Data Processor)} обеспечивает загрузку и актуализацию графа из внешних источников.
    \item \textbf{База данных (PostgreSQL)} хранит картографические данные OSM и топологию графа, а также занимается вычислением маршрутов.
\end{enumerate}

\begin{figure}[H]
    \centering
    \includemermaid[width=\linewidth, height=0.75\textheight, keepaspectratio]{01-system-arch}
    \caption{Функциональная схема архитектуры системы: взаимодействие компонентов через шлюз и очередь сообщений}
    \label{fig:system-arch}
\end{figure}

Определив общую архитектуру и стек технологий, необходимо формализовать математическую модель транспортной сети и разработать алгоритмы, которые позволят эффективно обрабатывать граф в рамках выбранной СУБД-ориентированной парадигмы.

\section{Математическая модель транспортной сети}

В основе навигационной системы лежит представление дорожной сети в виде графа $G = (V, E)$. Целевая функция поиска маршрута $P^*$ заключается в минимизации суммарной стоимости прохождения ребер:
\begin{equation}
P^* = \arg \min_{P \in \mathcal{P}_{st}} \sum_{e \in P} w(e, t),
\end{equation}
где $w(e, t)$ — весовая функция, зависящая от времени. Она рассчитывается как отношение длины участка $L_e$ к эффективной скорости $v_{eff}$, которая моделируется на основе диаграммы \textbf{Гриншилдса} (Greenshields Traffic Flow Model)\cite{trb2015fundamental}:
\begin{equation}
    v = v_f \left( 1 - \frac{k}{k_j} \right),
\end{equation}
где $v_f$ — скорость свободного потока, а $k_j$ — плотность затора.

Физическая модель данных, реализующая граф в реляционной СУБД, представлена на рисунке \ref{fig:er-schema}.

\begin{figure}[H]
    \centering
    \include{assets/diagrams/03-graph-schema}
    \caption{ER-диаграмма схемы базы данных: хранение графа $G(V,E)$ и метаданных OSM}
    \label{fig:er-schema}
\end{figure}

\section{Методика формирования графа дорог}

Для обеспечения корректности маршрутизации необходимо фильтровать исходные данные OpenStreetMap. В систему загружаются только объекты, имеющие значение для автомобильной навигации. Процесс фильтрации реализуется на языке Overpass QL и включает 4 этапа.

\begin{enumerate}
    \item \textbf{Белый список дорог (Whitelist).} Выбираются только линии с тегом \texttt{highway}, соответствующие асфальтированным дорогам (motorway, trunk, primary) и жилым улицам (residential). Пешеходные тропы и сервисные проезды исключаются.
    \item \textbf{Ограничения поворотов (Turn Restrictions).} Извлекаются отношения (relations) типа \texttt{restriction}, определяющие запреты маневров (например, only\_right\_turn).
    \item \textbf{Точечные барьеры.} Учитываются узлы с тегом \texttt{barrier} (шлагбаумы, ворота), которые могут блокировать проезд.
    \item \textbf{Ограничения доступа (Access Restrictions).} Исключаются частные территории (\texttt{access=private}) и выезды с парковок (\texttt{service=driveway}), так как они часто создают ложные "срезы" маршрутов через дворы.
\end{enumerate}

Для реализации данной фильтрации применяется стратегия \textbf{рекурсивного извлечения данных} (Recursive Extraction Strategy), обусловленная особенностями объектной модели OSM. Процесс формирования выборки $D_{raw}$ описывается последовательностью операций:

\begin{enumerate}
    \item \textbf{Селекция метаданных (\texttt{out body}).} На первом этапе выполняется поиск идентификаторов объектов (Way, Node, Relation), удовлетворяющих атрибутивным фильтрам. Геометрия на данном этапе игнорируется для минимизации трафика.
    \item \textbf{Рекурсивное разрешение зависимостей (\texttt{>}).} Поскольку линии (Ways) в OSM состоят из ссылок на узлы (Nodes), а отношения (Relations) — из ссылок на линии и точки, выполняется операция топологического спуска (recurse down). Это гарантирует целостность графа.
    \item \textbf{Генерация скелетной геометрии (\texttt{out skel qt}).} Для передачи данных используется компактный формат <<скелета>>, содержащий только координаты узлов. Модификатор \texttt{qt} (Quasi-Tile) оптимизирует порядок сортировки данных для потоковой обработки.
\end{enumerate}

Таким образом, граф $G$ формируется из подмножества объектов OSM, удовлетворяющих условиям фильтрации:
\begin{equation}
    E \subset \{ w \in \text{OSM}_{ways} \mid \text{tag}(w) \in \text{Whitelist} \land \text{access}(w) = \text{public} \}
\end{equation}

Детальная реализация фильтра на языке запросов Overpass QL приведена в главе \ref{chap:implementation}.


\subsection{Метод пространственного секционирования (Grid Partitioning)}
Для решения проблемы исчерпания памяти при построении топологии графа применен алгоритм Grid Partitioning. Исходный граф разбивается на тайлы, что позволяет строить узловую сеть (\texttt{pgr\_nodeNetwork}) независимо для каждого фрагмента $P \times P$\cite{zhu2015gridgraph}.

\subsection{Динамическое ограничение области поиска}
Для оптимизации алгоритма Дейкстры используется метод \textbf{Goal Pruning}. Поиск выполняется только внутри ограничивающего прямоугольника (Bounding Box), рассчитываемого динамически:
\begin{equation}
\delta(d) = \max \left(0.015^\circ, d \cdot 0.3 \right).
\label{eq:dynamic-bbox}
\end{equation}
Это позволяет отсекать до 90\% нерелевантных ребер\cite{sturtevant2015bounding}.

\subsection{Поиск альтернативных маршрутов}
Применяется метод \textbf{Iterative Penalty}. После нахождения $k$-го маршрута веса его ребер умножаются на штрафной коэффициент $\mu$, вынуждая алгоритм искать топологически отличные пути\cite{akgun2013penalty}.

\subsection{Анализ связности графа}
Для исключения изолированных подграфов ("островов"), из которых невозможен выезд на основную сеть, используется алгоритм Тарьяна поиска сильно связных компонент (SCC)\cite{tarjan2020algorithm}.

\begin{figure}[H]
    \centering
    \includegraphics[width=0.95\linewidth]{components_map}
    \caption{Карта связности графа. Выделены изолированные кластеры (острова), исключенные из маршрутизации.}
    \label{fig:components-map}
\end{figure}


\section{Выводы по первой главе}

В данной главе сформирован теоретический и архитектурный базис системы. Обоснован отказ от In-Memory движков (OSRM) в пользу Database-Centric архитектуры на базе PostgreSQL, что обеспечивает мгновенную реакцию на изменения дорожной обстановки. Математическая модель, основанная на диаграмме Гриншилдса, позволяет учитывать зависимость скорости от плотности потока. Внедренный комплекс алгоритмов (Grid Partitioning, Dynamic BBox, Iterative Penalty) решает проблемы масштабируемости и производительности, делая возможным построение динамических маршрутов на стандартном оборудовании.
