\section{Анализ задачи маршрутизации в динамической среде}

\subsection{Специфика маршрутизации в динамической среде}

Задача поиска оптимального маршрута в транспортной сети сводится к нахождению кратчайшего пути во взвешенном ориентированном графе:
\begin{equation}
    G = (V, E),
\end{equation}
\where{
    $V$ & множество вершин графа, соответствующих перекресткам и дорожным развязкам; \\
    $E$ & множество ориентированных ребер, представляющих дорожные сегменты; \\
    $w(e)$ & весовая функция ребра (стоимость проезда), которая в классической постановке считается константой.
}

Классический подход позволяет использовать эффективные алгоритмы предварительной обработки, такие как метод иерархий сокращений (Contraction Hierarchies) \cite{geisberger2008_researchgate}, сокращающий время запроса до миллисекунд.

Однако в условиях городской среды веса ребер $w(e, t)$ являются функцией времени, изменяясь под воздействием пробок, аварий или перекрытий дорог.

\subsection{Сравнительный анализ существующих решений}

Для выбора платформы был проведен сравнительный анализ существующих решений \cite{routexl2025comparison, gisops2018routing}. На современном рынке выделяются два основных класса систем.

\begin{enumerate}
    \item \textbf{Системы с предварительной обработкой в оперативной памяти} (OSRM, GraphHopper, Valhalla). Ориентированы на максимальную производительность поиска за счет сложного этапа предварительной подготовки данных. Граф компилируется в оптимизированные структуры данных, такие как иерархии сокращений или векторные тайлы.
    \item \textbf{СУБД-ориентированные решения} (pgRouting). Ориентированы на гибкость и работу с динамическими данными непосредственно в системе управления базами данных без предварительного построения индексов.
\end{enumerate}

В таблице \ref{tab:routing-engines-general} представлено сравнение архитектурных особенностей данных решений.

\begin{table}[H]
    \centering
    \caption{Сравнительный анализ архитектурных подходов к маршрутизации}
    \label{tab:routing-engines-general}
    \small
    \begin{tabularx}{\textwidth}{|p{0.18\linewidth}|X|X|X|}
        \hline
        \textbf{Решение} & \textbf{Алгоритм оптимизации} & \textbf{Механизм работы} & \textbf{Ограничения для динамических графов} \\ \hline
        \textbf{OSRM} & Метод иерархий сокращений (CH) / MLD & Строит иерархию сокращений во время предварительной обработки. & Любое изменение весов требует полной перестройки графа (до 1 часа для Москвы). \\ \hline
        \textbf{GraphHopper} & CH / Метод ориентиров (ALT) & Использует гибридный подход. Гибкий режим позволяет менять параметры, но без изменения топологии. & Высокое потребление оперативной памяти (Java Heap). Пересчет весов в реальном времени неэффективен. \\ \hline
        \textbf{Valhalla} & Иерархические тайлы & Граф разбит на фрагменты. \textbf{Динамический расчет стоимости} вычисляет веса на основе атрибутов фрагмента \cite{saki2022mapmatching}. & Обновление весов требует пересборки фрагментов и перезапуска сервиса. Максимальная частота обновления -- раз в неделю (исторические данные). \\ \hline
        \textbf{pgRouting} & \textbf{Отсутствует} (Алгоритм Дейкстры / A*) & Работает напрямую с топологией в таблицах PostGIS. Веса вычисляются SQL-запросом <<на лету>> \cite{crunchydata2026routing}. & Линейная зависимость времени поиска от размера графа. Требует высокой производительности дисковой подсистемы. \\ \hline
    \end{tabularx}
\end{table}

Как показывает анализ, решения, выполняющие вычисления в оперативной памяти (OSRM, GraphHopper, Valhalla), жертвуют возможностью мгновенного изменения весов ради производительности. В частности, система Valhalla, несмотря на заявленный механизм динамического расчета стоимости пути, для актуализации данных о трафике требует периодической пересборки графовых фрагментов и перезапуска сервиса маршрутизации: \textit{<<you would need to periodically update your tileset and restart your routing service to make the changes live>>} \cite{valhalla_issue4030}. Подобные архитектурные ограничения делают эти системы непригодными для задач мультиагентной навигации, где весовые коэффициенты ребер должны обновляться непрерывно по мере движения агентов.
% \textit{<<Вам потребуется периодически обновлять набор тайлов и перезапускать сервис маршрутизации для применения изменений>>} %

\section{Обоснование выбора СУБД-ориентированной архитектуры}

На основе анализа было принято решение использовать архитектуру, ориентированную на базы данных. Граф дорог хранится и обрабатывается непосредственно в реляционной СУБД PostgreSQL. Данный подход обладает рядом преимуществ для геоинформационных систем\cite{basargin2021pgrouting}:
\begin{enumerate}
    \item \textbf{Транзакционная целостность.} Маршруты строятся на основе согласованного снимка данных, обеспечиваемого механизмом многоверсионности (MVCC), даже если параллельно идут процессы обновления графа.
    \item \textbf{Единая точка истины.} Локационные и бизнес-данные (пробки, инциденты) находятся в одном хранилище, устраняя необходимость синхронизации с внешними поисковыми индексами\cite{percona2025postgis}.
\end{enumerate}

Расширение pgRouting, в свою очередь, обеспечивает атомарное изменение весов через стандартные SQL-операции (\texttt{UPDATE}), что критически важно для реализации динамической маршрутизации в условиях меняющегося трафика.

\section{Архитектура программного комплекса}

Система спроектирована по принципам сервис-ориентированной архитектуры, как показано на рисунке \ref{fig:system-arch}. Взаимодействие компонентов организовано следующим образом:
\begin{enumerate}
    \item \textbf{Шлюз доступа (Gateway)} принимает запросы клиентов и маршрутизирует их.
    \item \textbf{Сервис маршрутизации (Router)} выполняет SQL-запросы к базе данных для построения маршрутов.
    \item \textbf{Сервис обработки данных (Data Processor)} обеспечивает загрузку и актуализацию графа из внешних источников.
    \item \textbf{База данных (PostgreSQL)} хранит картографические данные OSM и топологию графа, а также занимается вычислением маршрутов.
\end{enumerate}

\begin{figure}[H]
    \centering
    \includemermaid[width=\linewidth, height=0.75\textheight, keepaspectratio]{01-system-arch}
    \caption{Функциональная схема архитектуры системы: взаимодействие компонентов через шлюз и очередь сообщений}
    \label{fig:system-arch}
\end{figure}

Определив общую архитектуру и стек технологий, необходимо формализовать математическую модель транспортной сети и разработать алгоритмы, которые позволят эффективно обрабатывать граф в рамках выбранной СУБД-ориентированной парадигмы.
