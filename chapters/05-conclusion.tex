\chapter*{ЗАКЛЮЧЕНИЕ}
\addcontentsline{toc}{chapter}{ЗАКЛЮЧЕНИЕ}

В ходе выполнения научно-исследовательской работы получены следующие основные результаты:

\begin{enumerate}
    \item Проведен анализ методов маршрутизации в динамических графах. Установлено, что существующие решения (OSRM, GraphHopper) не обеспечивают эффективное обновление графа в реальном времени (время перестроения индекса OSRM достигает 15 минут). Обоснован выбор архитектуры, ориентированной на базы данных, которая позволяет учитывать изменения дорожной ситуации (перекрытия, пробки) мгновенно, используя стандартные SQL-транзакции.
    \item Разработано математическое и алгоритмическое обеспечение на основе метода пространственного секционирования. Применение данного алгоритма позволило снизить потребление оперативной памяти при построении топологии графа с 12.4~ГБ до 370~МБ и сократить время построения с часов до 13 минут, что обеспечивает возможность обработки графов масштаба мегаполиса на ограниченных вычислительных ресурсах.
    \item Спроектирована и реализована микросервисная архитектура программного комплекса на базе стека технологий PostgreSQL, PostGIS, pgRouting и FastAPI. Реализован механизм MVCC, обеспечивающий неблокирующее чтение данных. Экспериментально подтверждено линейное масштабирование системы: при увеличении числа конкурентных клиентов до 12 время отклика сократилось с 6.2~с до 1.2~с (ускорение в 5.2 раза).
    \item Разработано кроссплатформенное клиентское приложение с использованием фреймворка Qt и библиотеки MapLibre GL. Реализован механизм динамической подгрузки векторных тайлов формата MVT, обеспечивающий плавную визуализацию карты с высокой частотой кадров (60~FPS) и интерактивное взаимодействие с пользователем.
    \item Проведено экспериментальное исследование характеристик качества маршрутизации.
    \begin{itemize}
        \item Медианный коэффициент извилистости составил $T \approx 1.36$, что подтверждает эффективность движения без лишних перепробегов.
        \item Индекс семантического разнообразия альтернатив (индекс Жаккарда) не превышает 3~\%, а индекс разнообразия путей (PDI) составляет более 0.93, что свидетельствует о топологической независимости предлагаемых маршрутов.
        \item Успешность построения маршрута для стандартных запросов ($N=2$) составляет 95~\%, что подтверждает надежность алгоритма в условиях реальной городской застройки.
    \end{itemize}
\end{enumerate}
