\chapter*{ЗАКЛЮЧЕНИЕ}
\addcontentsline{toc}{chapter}{ЗАКЛЮЧЕНИЕ}

В ходе выполнения курсовой работы была спроектирована и реализована распределенная система построения маршрутов на основе микросервисной архитектуры.

В результате исследования были получены следующие основные результаты:
\begin{enumerate}
    \item Обоснован выбор технологического стека \textbf{PostgreSQL + PostGIS + pgRouting}, обеспечивающего баланс между производительностью и гибкостью динамической маршрутизации.
    \item Разработан алгоритм \textbf{Grid Partitioning} (плиточная нарезка), позволивший снизить потребление оперативной памяти при построении графа Москвы с 12 ГБ до 370 МБ и сократить время сборки до 13 минут.
    \item Реализован механизм загрузки данных из OpenStreetMap, устойчивый к таймаутам и лимитам публичных API, обеспечивающий автоматическую актуализацию дорожной сети.
    \item Создан клиентский интерфейс на базе \textbf{PyQt5} и \textbf{MapLibre GL JS}, поддерживающий плавную визуализацию векторных тайлов и интерактивное взаимодействие с картой.
    \item Проведено нагрузочное тестирование, показавшее, что система способна обрабатывать запросы маршрутизации со средним временем отклика 6.2 секунды. Подтверждена эффективность параллельной обработки запросов.
\end{enumerate}

Полученные характеристики системы (время отклика 6 секунд) ограничивают ее применение в качестве навигатора реального времени для конечного пользователя, однако полностью удовлетворяют требованиям к системам логистического планирования и диспетчеризации, где ключевым фактором является возможность учета сложных динамических ограничений, невозможных в статических графовых движках.
