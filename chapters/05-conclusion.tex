\chapter*{ЗАКЛЮЧЕНИЕ}
\addcontentsline{toc}{chapter}{ЗАКЛЮЧЕНИЕ}

Выполненная выпускная квалификационная работа охватывает цикл проектирования и реализации высокопроизводительной навигационной системы, оптимизированной для работы в динамической городской среде. В ходе исследования был осуществлен переход от теоретического анализа задач маршрутизации к построению распределенного программного комплекса, способного адаптироваться к изменяющимся условиям дорожной сети в режиме реального времени.

На начальном этапе был проведен глубокий анализ существующих решений (OSRM, Valhalla, GraphHopper). Обоснован отказ от использования статических иерархий сокращений (CH) в пользу \textit{Database-Centric} архитектуры на базе связки PostgreSQL, PostGIS и pgRouting. Это решение позволило устранить критический недостаток современных систем — длительное (до 60 минут) время перестройки графа, обеспечив возможность мгновенного обновления весов ребер через стандартные транзакционные механизмы СУБД.

В рамках разработки математического обеспечения был предложен и внедрен метод пространственного секционирования графа (Grid Partitioning). Применение данного подхода в сочетании с алгоритмом \texttt{ST\_Subdivide} позволило решить фундаментальную проблему исчерпания оперативной памяти при построении топологии сверхбольших графов. В результате потребление ресурсов RAM было снижено с 12.4 ГБ до 370 МБ, что обеспечило стабильную работу системы на вычислительных мощностях среднего уровня.

Программная реализация системы выполнена с использованием современного стека технологий: асинхронного фреймворка FastAPI, драйвера \texttt{asyncpg} и кроссплатформенной среды Qt. Реализованный Data Processor обеспечил поддержку протокола «ленивой» загрузки данных из OpenStreetMap и динамическую генерацию векторных тайлов (On-Demand MVT) с учетом уровней детализации (LOD). Разработка гибридного клиентского приложения позволила объединить высокую скорость WebGL-рендеринга с богатыми возможностями десктопного интерфейса.

Финальное экспериментальное исследование на графе Московской агломерации подтвердило эффективность разработанных методов. Достигнутая латентность запроса (2.5 с при прогретом кэше) при практически линейной вертикальной масштабируемости (благодаря архитектуре MVCC) доказывает применимость комплекса для задач мягкого реального времени. Анализ семантического разнообразия альтернативных путей (индекс Жаккарда составил в среднем 0.0278) подтвердил корректную работу метода итеративных штрафов, обеспечивающего эффективную деконцентрацию транспортных потоков.

Таким образом, работа демонстрирует последовательное решение комплекса научно-технических задач: от оптимизации хранения данных до построения высокоуровневого пользовательского интерфейса. Разработанная система может быть использована как фундамент для построения мультиагентных транспортных систем и интеллектуальных логистических комплексов нового поколения.
