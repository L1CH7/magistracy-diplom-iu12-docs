\section{Методология экспериментального исследования}

\subsection{Характеристика тестового полигона}

В качестве полигона для испытаний был использован граф дорожной сети Московской агломерации, включающий территорию МКАД и пригородную зону. Выбор данной области обусловлен высокой плотностью графа и наличием сложных топологических структур (многоуровневые развязки, кольцевые магистрали, мосты), создающих максимальную вычислительную нагрузку на алгоритмы обхода.

Параметры тестовой области:
\begin{itemize}
    \item \textbf{долгота} от $37.30^{\circ}$E до $37.90^{\circ}$E;
    \item \textbf{широта} от $55.45^{\circ}$N до $56.00^{\circ}$N.
\end{itemize}

\subsection{Аппаратная конфигурация тестового стенда}

Характеристики аппаратного обеспечения, использованного при проведении экспериментов:
\begin{itemize}
    \item центральный процессор AMD Ryzen 5 5600H (архитектура Zen 3, 6 физических ядер, 12 потоков, базовая частота 3,3~ГГц);
    \item оперативная память 16~ГБ DDR4-3200 (двухканальный режим), из которых 4~ГБ выделено под буферный пул PostgreSQL (\texttt{shared\_buffers});
    \item дисковая подсистема на базе NVMe SSD (M.2, PCIe 3.0 x4) со скоростью последовательного чтения до 3500~МБ/с;
    \item графический ускоритель NVIDIA GeForce RTX 3060 Laptop (6~ГБ GDDR6), задействованный для рендеринга векторных тайлов;
    \item операционная система Linux Arch (ядро 6.17.5-arch1-1) с файловой системой \texttt{ext4}.
\end{itemize}

\subsection{Инструментарий нагрузочного тестирования}

Для проведения экспериментов был разработан специализированный программный комплекс на языке Python, архитектура которого обеспечивает минимизацию накладных расходов на генерацию нагрузки. Программный комплекс, разработанный для проведения экспериментов, включает ряд функциональных компонентов.
\begin{itemize}
    \item Модуль оркестрации тестирования~-- фреймворк \texttt{pytest} с модулем \texttt{pytest-xdist} для параллельного запуска тестов. Использовалось 12 независимых воркеров (процессов), что соответствует числу логических ядер процессора, для создания максимальной конкурентной нагрузки.
    \item Сетевой транспортный модуль~-- асинхронный HTTP-клиент \texttt{httpx}, настроенный в режиме пулинга соединений. Повторное использование TCP-соединений исключает накладные расходы на Handshake и TLS Negotiation при каждом запросе, позволяя измерить <<чистое>> время обработки запроса на сервере.
    \item Сбор и анализ метрик~-- агрегация результатов (Latency, Throughput, Error Rate) производится в формате CSV с последующей статистической обработкой в библиотеке \texttt{pandas}.
\end{itemize}

\subsection{Стратегии выборки данных}

Для исключения эффекта <<горячего кэша>> и обеспечения статистической репрезентативности результатов, в модуле \texttt{utils.py} были реализованы две различные стратегии генерации пар точек старта и финиша.

Для нагрузочных тестов применяется равномерная сеточная выборка. Координаты точек привязываются к узлам регулярной сетки с шагом $\approx 0.05^{\circ}$ (около 5 км) с использованием функции \texttt{ST\_SnapToGrid}. Данный подход гарантирует равномерное покрытие всей площади исследуемого полигона и предотвращает локализацию запросов в одном районе, что могло бы искусственно завысить показатели производительности за счет кэширования страниц БД.

Для метрической оценки качества маршрутизации (коэффициенты извилистости и Жаккарда) используется стратегия удаленных точек. Генератор выбирает случайные пары точек с условием, что евклидово расстояние между ними превышает пороговое значение:
\begin{equation}
L_{euclid} > 4000 \text{ м}.
\end{equation}

Фильтрация коротких маршрутов необходима для исключения тривиальных внутриквартальных путей, где альтернативные варианты проезда физически отсутствуют. Это позволяет сфокусировать тестирование на магистральных направлениях, где эффективность алгоритма маршрутизации критически важна.
