\section{Оценка качества маршрутизации и надежности}

Помимо скоростных характеристик, критически важным аспектом является воспринимаемое качество маршрутов — их логичность, безопасность и предсказуемость.
Для количественной оценки качества использовались интегральные метрики извилистости и семантического разнообразия.

\subsection{Коэффициент извилистости (Tortuosity)}

Метрика извилистости $T$ определяется как отношение длины найденного маршрута к евклидову расстоянию между точками старта и финиша:
\begin{equation}
T = \frac{L_{route}}{L_{euclid}}
\end{equation}

Анализ гистограммы распределения (см. рис.~\ref{fig:tortuosity-hist}) показал, что медианное значение составляет \textbf{$T \approx 1.36$}.

\begin{figure}[H]
    \centering
    \includegraphics[width=0.8\linewidth]{assets/images/latency_tortuosity_hist}
    \caption{Гистограмма коэффициента извилистости (Tortuosity) для найденных маршрутов}
    \label{fig:tortuosity-hist}
\end{figure}

Это указывает на высокую эффективность алгоритма Дейкстры с динамическим ограничением области поиска на участках с регулярной планировкой (Grid Street Plan).
Однако хвост распределения ($T > 2.5$) составляют маршруты, пролегающие через водные преграды (реки, каналы) при отсутствии мостов в прямой видимости. В таких случаях алгоритм вынужден совершать значительные объезды, что является корректным поведением с точки зрения топологии, но может восприниматься пользователем как субоптимальное решение.

\subsection{Семантическое разнообразие маршрутов}

Для оценки эффективности метода итеративных штрафов (Iterative Penalty Dijkstra) при поиске альтернативных путей была применена метрика пространственного разнообразия — индекс Жаккарда (Jaccard Index). Данный коэффициент вычисляется как отношение числа общих ребер между основным ($R_1$) и альтернативным ($R_2$) маршрутами к общему числу уникальных ребер:
\begin{equation}
J(R_1, R_2) = \frac{|E(R_1) \cap E(R_2)|}{|E(R_1) \cup E(R_2)|}
\end{equation}

Экспериментальное исследование проводилось с использованием стратегии выборки «Distant Points»: генерировались пары узлов на расстоянии не менее 4 км друг от друга (средняя дистанция в выборке составила 35 км). Это позволило оценить работу штрафных коэффициентов при прохождении через ключевые транспортные артерии и узлы (мосты, развязки), где альтернативные пути ограничены топологией города.

\begin{figure}[H]
    \centering
    \includegraphics[width=0.8\linewidth]{assets/images/jaccard_vs_distance}
    \caption{Зависимость коэффициента Жаккарда от расстояния между точками (Selection Strategy: Distant Points)}
    \label{fig:jaccard-vs-dist}
\end{figure}

\begin{figure}[H]
    \centering
    \includegraphics[width=0.8\linewidth]{assets/images/pdi_distribution}
    \caption{Распределение индекса разнообразия путей (Path Diversity Index)}
    \label{fig:pdi-dist}
\end{figure}

Интерпретация визуализированных данных (рисунки \ref{fig:jaccard-vs-dist}, \ref{fig:pdi-dist}) и анализ полученных метрик позволяют сделать следующие выводы:
\begin{itemize}
    \item \textbf{Эволюция метрик и адаптивность к топологии.} Средний коэффициент Жаккарда составил $J \approx 0.0278$. По сравнению с первичными тестами на коротких дистанциях, рост этого показателя подтверждает переход системы от анализа разреженных участков к реальной городской среде. Тем не менее, среднее совпадение маршрутов менее чем на 3\% является выдающимся результатом для задач распределения потоков в реальном времени.
    \item \textbf{Эффект «бутылочного горлышка».} На графике рассеяния (рис.~\ref{fig:jaccard-vs-dist}) зафиксировано отчетливое «облако» значений в диапазоне $J \in [0.02, 0.10]$ и локальные пики до $J \approx 0.24$. Данные аномалии строго соответствуют маршрутам, пересекающим естественные топологические преграды (реки, железнодорожные пути). В этих точках алгоритм успешно балансирует между поиском новых путей и здравым смыслом: при штрафе $\mu = 5.0$ система выбирает частичное совпадение на мосту, так как объезд через следующую переправу увеличил бы суммарный путь в несколько раз.
    \item \textbf{Стабильность индекса разнообразия (PDI).} Несмотря на топологические ограничения, индекс PDI стабильно превышает 0.93. Это означает, что 93\% всех ребер в наборе из 5 найденных маршрутов являются абсолютно уникальными. Данный результат подтверждает робастность алгоритма и его применимость для мультиагентных систем (МАС), где требуется высокая вариативность путей для эффективной балансировки нагрузки.
\end{itemize}

ИТОГ: Тестирование на дальних дистанциях полностью подтвердило робастность алгоритма. Полученная картина распределения (низкая корреляция сходства с дистанцией при сохранении высокого разнообразия) является надежным доказательством качества разработанного метода итеративных штрафов.

\subsection{Надежность и вероятность построения (Yield Rate)}



Тепловая карта успешности построения маршрутов (рисунок~\ref{fig:yield-heatmap}) выявила фундаментальную проблему связности графа.

\begin{figure}[H]
    \centering
    \includegraphics[width=0.9\linewidth]{assets/images/throughput_yield_heatmap}
    \caption{Тепловая карта успешности построения (Yield Rate) в зависимости от числа агентов и длины маршрута. Красные зоны соответствуют высокой вероятности отказа (NoRouteFound).}
    \label{fig:yield-heatmap}
\end{figure}

\begin{itemize}
    \item Для задачи поиска одного пути ($K=1$) успешность (Yield Rate) составила \textbf{100\%}, за исключением случаев, когда точка старта попадала в изолированный граф (Island Problem) — например, закрытую территорию завода или военный объект без выезда на дороги общего пользования.
    \item При увеличении числа требуемых альтернатив ($N=5, K=5$) успешность падала до \textbf{41\%}.
\end{itemize}

Анализ отказов показал следующее распределение: в 14.8\% случаев возникала абсолютная ошибка \texttt{NoRouteFound} (невозможность построить даже первый маршрут из-за старта в изолированной зоне). Оставшиеся 44.2\% запросов завершались частичным успехом: система успешно находила от 1 до 4 альтернатив, после чего исчерпывала топологические возможности локальной дорожной сети (например, упиралась в безальтернативный выезд из района). Данное поведение подтверждает корректность работы метода итеративных штрафов, который честно сообщает о невозможности найти семантически независимый путь, вместо того чтобы возвращать пользователю идентичные маршруты.
