\section{Оценка качества маршрутизации и надежности} \label{sec:quality}

Помимо скоростных характеристик, критически важным аспектом является воспринимаемое качество маршрутов -- их логичность, безопасность и предсказуемость.
Для количественной оценки качества использовались интегральные метрики извилистости и семантического разнообразия.

\subsection{Коэффициент извилистости}

Метрика извилистости $T$ определяется как отношение длины найденного маршрута к евклидову расстоянию между точками старта и финиша:
\begin{equation}
T = \frac{L_{route}}{L_{euclid}},
\end{equation}
\where{
    $L_{route}$ & длина найденного маршрута; \\
    $L_{euclid}$ & евклидово расстояние по прямой.
}

Гистограмма распределения значений коэффициента извилистости представлена на рисунке \ref{fig:tortuosity-hist}.

\begin{figure}[H]
    \centering
    \includegraphics[width=0.8\linewidth]{assets/images/latency_tortuosity_hist}
    \caption{Гистограмма коэффициента извилистости (Tortuosity) для найденных маршрутов}
    \label{fig:tortuosity-hist}
\end{figure}

Анализ гистограммы показал, что медианное значение составляет \textbf{$T \approx 1.36$}. Распределение имеет логнормальное характер, что позволяет выделить две группы маршрутов.

\begin{enumerate}
    \item Основной пик ($T \in [1.2, 1.7]$) соответствует эффективному движению по уличной сетке регулярной планировки. Близость к единице подтверждает, что механизм адаптивного окна корректно ограничивает область поиска, не позволяя алгоритму Dijkstra уходить в глубокий перебор.
    \item Длинный хвост ($T > 2.5$) не является аномалией, а отражает корректную работу алгоритма в условиях естественных преград, например, изгибы Москвы-реки, железнодорожные пути. В таких ситуациях прямой путь геометрически невозможен, поэтому система строит оптимальный топологический объезд.
\end{enumerate}

\subsection{Семантическое разнообразие маршрутов}

Для оценки эффективности метода итеративных штрафов при поиске альтернативных путей использовалась выборка случайной пары точек на расстоянии более $4$ км. 

Метрикой оценки был выбран индекс Жаккарда, который вычисляется по формуле:

\begin{equation}
J(R_1, R_2) = \frac{|E(R_1) \cap E(R_2)|}{|E(R_1) \cup E(R_2)|},
\end{equation}
\where{
    $E(R_1)$ & множество ребер основного маршрута; \\
    $E(R_2)$ & множество ребер альтернативного маршрута.
}
Дополнительно для оценки топологической независимости альтернатив использовался индекс разнообразия путей (Path Diversity Index), рассчитываемый по формуле:

\begin{equation}
PDI = \frac{\left| \bigcup_{i=1}^{K} E(R_i) \right|}{\sum_{i=1}^{K} \left| E(R_i) \right|},
\end{equation}
\where{
    $E(R_i)$ & множество ребер $i$-го маршрута; \\
    $K$ & количество найденных альтернативных маршрутов.
}

Значение $PDI \to 1$ указывает на то, что маршруты практически не имеют общих участков (максимальное разнообразие), тогда как $PDI \to 1/K$ свидетельствует о полном совпадении путей.

Для визуализации результатов эксперимента были построены два графика. На рисунке \ref{fig:jaccard-vs-dist} показана зависимость коэффициента Жаккарда от расстояния между точками, а на рисунке \ref{fig:pdi-dist} -- распределение индекса разнообразия путей.

\begin{figure}[H]
    \centering
    \includegraphics[width=0.8\linewidth]{assets/images/jaccard_vs_distance}
    \caption{Зависимость коэффициента Жаккарда от расстояния между точками}
    \label{fig:jaccard-vs-dist}
\end{figure}

\begin{figure}[H]
    \centering
    \includegraphics[width=0.8\linewidth]{assets/images/pdi_distribution}
    \caption{Распределение индекса разнообразия путей}
    \label{fig:pdi-dist}
\end{figure}

Интерпретация полученных данных позволяет сделать следующие выводы:

\begin{itemize}
    \item средний коэффициент Жаккарда $J \approx 0.0278$ (менее 3~\%).
    \item индекс разнообразия путей (PDI) более $0.93$.
\end{itemize}

Низкий процент перекрытия доказывает, что алгоритм находит топологически независимые магистрали (например, предлагая выбор между Ленинградским и Дмитровским шоссе), а не локальные вариации одного и того же пути. Локальные всплески сходства на графике объясняются наличием <<бутылочных горлышек>> (мостов и тоннелей), где физически отсутствуют альтернативные маршруты.

\subsection{Коэффициент успешности построения}

Визуализация зависимости вероятности успешного построения маршрута от его длины и количества агентов приведена на рисунке \ref{fig:yield-heatmap}. Как видно из графика, успешность построения маршрута практически не зависит от количества запрашиваемых альтернатив $K$.

\begin{figure}[H]
    \centering
    \includegraphics[width=0.9\linewidth]{assets/images/throughput_yield_heatmap}
    \caption{Тепловая карта успешности построения }
    \label{fig:yield-heatmap}
\end{figure}

Статистический анализ успешности построения маршрутов позволил выявить ряд закономерностей:

\begin{itemize}
    \item показатель успешности при $N=2$ составляет \textbf{95~\%};
    \item при маршруте с $N=5$ снижается до \textbf{69.2~\%};
    \item при варьировании числа альтернатив ($K$) показатель нахождения маршрута остается стабильным.
\end{itemize}

В среднем, уровень отказов и частичных результатов составил 14.8~\%. Анализ причин показал, что основные факторы отказов топологические, а не алгоритмические.

\begin{enumerate}
    \item Граничные эффекты~-- тестовый полигон ограничен, потому маршруты на периферии часто оптимально проходят через область (за границами BBox), что приводит к невозможности построения пути внутри полигона.
    \item Компромисс точности~-- использован агрессивный коэффициент расширения динамического окна $\delta(d) = 0.3 \cdot d$. Согласно исследованию \cite{zhou2017bounding}, это позволяет значительно ускорить поиск, но может отсекать валидные маршруты, требующие глубокого объезда.
    \item Проблема связности~-- вероятность попадания одной из точек старта/финиша в изолированный подграф (на закрытую территорию) растет экспоненциально с увеличением числа агентов $N$. Проблема несвязных компонент графа характерна для реальных данных OSM и требует процедур топологической фильтрации (прунинга) \cite{opentripplanner2023pruning}. Данная операция заключается в выделении максимальной компоненты связности дорожной сети и исключении изолированных участков, попадание на которые делает невозможным построение маршрута.
\end{enumerate}

Таким образом, экспериментальное исследование подтвердило высокую надежность разработанного алгоритмического обеспечения. Общий показатель успешности построения маршрутов превышает 85~\%.