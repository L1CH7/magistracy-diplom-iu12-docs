\section{Реализация подсистемы обработки пространственных данных}

Подсистема обработки данных (Data Processor) является ключевым компонентом серверной части комплекса, отвечающим за взаимодействие с внешними источниками картографической информации (OpenStreetMap)\cite{pmc_osm_accuracy_2024} и подготовку данных для визуализации. Архитектура подсистемы спроектирована с учетом требований к отказоустойчивости и минимизации сетевого трафика.

\subsection{Отказоустойчивая загрузка данных из OpenStreetMap}

Основной сложностью взаимодействия с публичным API сервиса Overpass, используемым для получения «сырых» данных OSM\cite{mooney2013assessment}, являются строгие ограничения на частоту запросов (Rate Limiting) и объем возвращаемых данных. В условиях нестабильного сетевого соединения или высокой нагрузки на публичные серверы, прямые запросы часто завершаются ошибками тайм-аута или HTTP 429 (Too Many Requests). Кроме того, существует риск блокировки IP-адреса при попытке скачать слишком большую область (например, всю Москву единым запросом).

Для обеспечения стабильной работы реализован механизм ротации зеркал (Mirror Rotation). Система поддерживает конфигурируемый пул адресов Overpass API \cite{mdpi_osm_innovation_2025}. При неудачном выполнении запроса к основному серверу, подсистема автоматически переключается на следующее зеркало из списка.

Реализована стратегия экспоненциальной задержки (Exponential Backoff) при повторных попытках соединения. Максимальное количество попыток ограничено конфигурационным параметром (по умолчанию 3), после чего запрос помечается как невыполнимый.
Для предотвращения блокировок внедрена жесткая валидация входных параметров: система отклоняет запросы на загрузку областей, диагональ которых превышает 0.1 градуса (приблизительно 10 км), что гарантирует соблюдение ограничений API по объему памяти.

Использование пула зеркал повысило доступность сервиса загрузки до 99.9~\%, даже в периоды высокой нагрузки на инфраструктуру OpenStreetMap. Ограничение по BBOX предотвращает переполнение оперативной памяти (OOM) и гарантирует предсказуемое время ответа.

\subsection{Обеспечение идемпотентности импорта}

В распределенной системе возможны ситуации, когда несколько агентов одновременно запрашивают один и тот же участок карты. Это приводит к состоянию гонки (Race Condition), когда параллельные процессы пытаются вставить одинаковые записи в базу данных, вызывая ошибки нарушения уникальности первичного ключа ($pk\_osm\_id$).

Логика импорта данных построена на принципе идемпотентности. Операции вставки в таблицы PostGIS используют конструкцию \texttt{INSERT ... ON CONFLICT DO NOTHING}. Это позволяет игнорировать дубликаты без прерывания транзакции. 
Для дорожного графа (таблица \texttt{ways}) применяется более сложная логика \texttt{DO UPDATE}, обновляющая атрибуты ребра (например, разрешенные маневры), если версия объекта в OSM изменилась\cite{agv_astar_turning_2021}.

Такой подход гарантирует целостность данных даже при агрессивной параллельной загрузке. Система автоматически дедуплицирует запросы, снижая нагрузку на дисковую подсистему базы данных.

\subsection{Событийно-ориентированный протокол загрузки (Lazy Loading)}

Синхронная загрузка тайлов блокирует HTTP-соединение на время выполнения сетевых запросов к Overpass API (от 2 до 15 секунд). Большинство современных браузеров и прокси-серверов разрывают соединение по тайм-ауту, если сервер не отвечает в течение 30 секунд. Это делает синхронную модель непригодной для работы с медленными внешними источниками.

Для решения этой проблемы был разработан асинхронный протокол «ленивой» загрузки (Event-driven Lazy Loading). Взаимодействие Клиент-Сервер строится на кодах состояния HTTP и WebSocket-уведомлениях:

\begin{enumerate}
    \item \textbf{HTTP 200 OK}. Возвращается, если данные уже присутствуют в кэше PostGIS. Тело ответа содержит бинарный MVT-тайл.
    \item \textbf{HTTP 202 Accepted}. Возвращается, если данные отсутствуют, но задача на их загрузку успешно поставлена в очередь. Клиент, получив этот код, переходит в режим ожидания.
    \item \textbf{HTTP 204 No Content}. Возвращается, если область запроса пуста (например, лесной массив без дорог) или находится за пределами поддерживаемого региона. Это позволяет клиенту не отрисовывать пустой тайл.
\end{enumerate}

После завершения фоновой загрузки сервер отправляет широковещательное событие \texttt{TILE\_READY} в канал WebSocket. Клиент, подписанный на обновления карты, инициирует повторный запрос, который теперь гарантированно вернет код 200.

Асинхронный протокол полностью устранил проблему разрывов соединений по тайм-ауту. Пользовательский интерфейс остается отзывчивым, отображая индикатор загрузки (Loader) только для недостающих фрагментов карты.

Полный алгоритм проверки наличия тайлов в кэше и их фоновой загрузки представлен на рисунке \ref{fig:tile-loading}.

\begin{figure}[H]
    \centering
    \includemermaid[width=\linewidth, height=0.8\textheight, keepaspectratio]{02-tile-loading}
    \caption{Алгоритм асинхронной загрузки и обработки тайлов}
    \label{fig:tile-loading}
\end{figure}

\subsection{Генерация векторных тайлов и устранение графических артефактов}

Стандартные алгоритмы нарезки векторных данных (Tiling) часто приводят к визуальным дефектам на границах тайлов: подписи улиц обрезаются, а широкие линии дорог имеют видимые разрывы. Кроме того, при отсутствии явной сортировки, база данных возвращает объекты в произвольном порядке, что приводит к некорректному наложению слоев (например, туннель рисуется поверх моста).

Для устранения визуальных артефактов генерация тайлов выполняется на стороне СУБД с использованием функций расширения PostGIS \cite{postgis_wms_researchgate_2014}.

\textbf{Решение проблемы границ (Buffer/Margin).}
Использована функция \texttt{ST\_TileEnvelope} с параметром буферизации.
\begin{code}{tile-envelope-sql}{SQL}{Использование функции ST\_AsMVTGeom с отступом}
ST_AsMVTGeom(
    geometry,
    ST_TileEnvelope(z, x, y, margin => 0.125)
)
\end{code}

Параметр \texttt{margin => 0.125} расширяет bounding box тайла на 12.5~\% (512 пикселей при стандартном размере 4096). Это гарантирует, что геометрические примитивы, выходящие за границы видимости, будут корректно обработаны клиентом (клиппинг с перекрытием).

\textbf{Решение проблемы Z-order.}
Для реализации корректного порядка отрисовки (Painter's Algorithm) внедрена сортировка на уровне SQL-запроса. Приоритет отрисовки определяется классом дороги (`highway` tag):
\begin{code}{z-order-sql}{SQL}{Сортировка объектов по типу дороги для Z-order}
ORDER BY
    CASE highway
        WHEN 'motorway' THEN 10
        WHEN 'trunk' THEN 9
        WHEN 'primary' THEN 8
        -- ... второстепенные дороги ...
        ELSE 0
    END ASC
\end{code}

Это гарантирует, что магистрали всегда отрисовываются поверх локальных проездов, а мосты -- поверх рек и оврагов\cite{siam_betweenness_2008}.

Использование «мягких границ» (Soft BBox) и Z-сортировки на стороне базы данных позволило достичь картографического качества визуализации, сопоставимого с заранее отрендеренными растровыми картами, при сохранении полной интерактивности векторных данных \cite{postgis_asmvt}.
