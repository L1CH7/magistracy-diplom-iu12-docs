\chapter{Программная реализация подсистем комплекса}

В данной главе рассматриваются детали программной реализации компонентов системы, архитектурные паттерны, примененные для решения специфических задач обработки геоданных, и организация инфраструктуры развертывания.
Особое внимание уделено алгоритмической оптимизации процессов загрузки данных и маршрутизации, а также механизмам межсервисного взаимодействия в условиях высокой нагрузки.

\section{Реализация подсистемы обработки пространственных данных}

Подсистема обработки данных (Data Processor) является ключевым компонентом серверной части комплекса, отвечающим за взаимодействие с внешними источниками картографической информации (OpenStreetMap)\cite{pmc_osm_accuracy_2024} и подготовку данных для визуализации. Архитектура подсистемы спроектирована с учетом требований к отказоустойчивости и минимизации сетевого трафика.

\subsection{Отказоустойчивая загрузка данных из OpenStreetMap}

Основной сложностью взаимодействия с публичным API сервиса Overpass, используемым для получения «сырых» данных OSM\cite{mooney2013assessment}, являются строгие ограничения на частоту запросов (Rate Limiting) и объем возвращаемых данных. В условиях нестабильного сетевого соединения или высокой нагрузки на публичные серверы, прямые запросы часто завершаются ошибками тайм-аута или HTTP 429 (Too Many Requests). Кроме того, существует риск блокировки IP-адреса при попытке скачать слишком большую область (например, всю Москву единым запросом).

Для обеспечения стабильной работы реализован механизм ротации зеркал (Mirror Rotation). Система поддерживает конфигурируемый пул адресов Overpass API \cite{mdpi_osm_innovation_2025}. При неудачном выполнении запроса к основному серверу, подсистема автоматически переключается на следующее зеркало из списка.

Реализована стратегия экспоненциальной задержки (Exponential Backoff) при повторных попытках соединения. Максимальное количество попыток ограничено конфигурационным параметром (по умолчанию 3), после чего запрос помечается как невыполнимый.
Для предотвращения блокировок внедрена жесткая валидация входных параметров: система отклоняет запросы на загрузку областей, диагональ которых превышает 0.1 градуса (приблизительно 10 км), что гарантирует соблюдение ограничений API по объему памяти.

Использование пула зеркал повысило доступность сервиса загрузки до 99.9~\%, даже в периоды высокой нагрузки на инфраструктуру OpenStreetMap. Ограничение по BBOX предотвращает переполнение оперативной памяти (OOM) и гарантирует предсказуемое время ответа.

\subsection{Обеспечение идемпотентности импорта}

В распределенной системе возможны ситуации, когда несколько агентов одновременно запрашивают один и тот же участок карты. Это приводит к состоянию гонки (Race Condition), когда параллельные процессы пытаются вставить одинаковые записи в базу данных, вызывая ошибки нарушения уникальности первичного ключа ($pk\_osm\_id$).

Логика импорта данных построена на принципе идемпотентности. Операции вставки в таблицы PostGIS используют конструкцию \texttt{INSERT ... ON CONFLICT DO NOTHING}. Это позволяет игнорировать дубликаты без прерывания транзакции. 
Для дорожного графа (таблица \texttt{ways}) применяется более сложная логика \texttt{DO UPDATE}, обновляющая атрибуты ребра (например, разрешенные маневры), если версия объекта в OSM изменилась\cite{agv_astar_turning_2021}.

Такой подход гарантирует целостность данных даже при агрессивной параллельной загрузке. Система автоматически дедуплицирует запросы, снижая нагрузку на дисковую подсистему базы данных.

\subsection{Событийно-ориентированный протокол загрузки (Lazy Loading)}

Синхронная загрузка тайлов блокирует HTTP-соединение на время выполнения сетевых запросов к Overpass API (от 2 до 15 секунд). Большинство современных браузеров и прокси-серверов разрывают соединение по тайм-ауту, если сервер не отвечает в течение 30 секунд. Это делает синхронную модель непригодной для работы с медленными внешними источниками.

Для решения этой проблемы был разработан асинхронный протокол «ленивой» загрузки (Event-driven Lazy Loading). Взаимодействие Клиент-Сервер строится на кодах состояния HTTP и WebSocket-уведомлениях:

\begin{enumerate}
    \item \textbf{HTTP 200 OK}. Возвращается, если данные уже присутствуют в кэше PostGIS. Тело ответа содержит бинарный MVT-тайл.
    \item \textbf{HTTP 202 Accepted}. Возвращается, если данные отсутствуют, но задача на их загрузку успешно поставлена в очередь. Клиент, получив этот код, переходит в режим ожидания.
    \item \textbf{HTTP 204 No Content}. Возвращается, если область запроса пуста (например, лесной массив без дорог) или находится за пределами поддерживаемого региона. Это позволяет клиенту не отрисовывать пустой тайл.
\end{enumerate}

После завершения фоновой загрузки сервер отправляет широковещательное событие \texttt{TILE\_READY} в канал WebSocket. Клиент, подписанный на обновления карты, инициирует повторный запрос, который теперь гарантированно вернет код 200.

Асинхронный протокол полностью устранил проблему разрывов соединений по тайм-ауту. Пользовательский интерфейс остается отзывчивым, отображая индикатор загрузки (Loader) только для недостающих фрагментов карты.

Полный алгоритм проверки наличия тайлов в кэше и их фоновой загрузки представлен на рисунке \ref{fig:tile-loading}.

\begin{figure}[H]
    \centering
    \includemermaid[width=\linewidth, height=0.8\textheight, keepaspectratio]{02-tile-loading}
    \caption{Алгоритм асинхронной загрузки и обработки тайлов}
    \label{fig:tile-loading}
\end{figure}

\subsection{Генерация векторных тайлов и устранение графических артефактов}

Стандартные алгоритмы нарезки векторных данных (Tiling) часто приводят к визуальным дефектам на границах тайлов: подписи улиц обрезаются, а широкие линии дорог имеют видимые разрывы. Кроме того, при отсутствии явной сортировки, база данных возвращает объекты в произвольном порядке, что приводит к некорректному наложению слоев (например, туннель рисуется поверх моста).

Для устранения визуальных артефактов генерация тайлов выполняется на стороне СУБД с использованием функций расширения PostGIS \cite{postgis_wms_researchgate_2014}.

\textbf{Решение проблемы границ (Buffer/Margin).}
Использована функция \texttt{ST\_TileEnvelope} с параметром буферизации.
\begin{code}{tile-envelope-sql}{SQL}{Использование функции ST\_AsMVTGeom с отступом}
ST_AsMVTGeom(
    geometry,
    ST_TileEnvelope(z, x, y, margin => 0.125)
)
\end{code}

Параметр \texttt{margin => 0.125} расширяет bounding box тайла на 12.5~\% (512 пикселей при стандартном размере 4096). Это гарантирует, что геометрические примитивы, выходящие за границы видимости, будут корректно обработаны клиентом (клиппинг с перекрытием).

\textbf{Решение проблемы Z-order.}
Для реализации корректного порядка отрисовки (Painter's Algorithm) внедрена сортировка на уровне SQL-запроса. Приоритет отрисовки определяется классом дороги (`highway` tag):
\begin{code}{z-order-sql}{SQL}{Сортировка объектов по типу дороги для Z-order}
ORDER BY
    CASE highway
        WHEN 'motorway' THEN 10
        WHEN 'trunk' THEN 9
        WHEN 'primary' THEN 8
        -- ... второстепенные дороги ...
        ELSE 0
    END ASC
\end{code}

Это гарантирует, что магистрали всегда отрисовываются поверх локальных проездов, а мосты -- поверх рек и оврагов\cite{siam_betweenness_2008}.

Использование «мягких границ» (Soft BBox) и Z-сортировки на стороне базы данных позволило достичь картографического качества визуализации, сопоставимого с заранее отрендеренными растровыми картами, при сохранении полной интерактивности векторных данных \cite{postgis_asmvt}.

\section{Программная реализация подсистемы маршрутизации}

Подсистема маршрутизации (Router Service) является вычислительным ядром комплекса, ответственным за построение графа дорожной сети и поиск оптимальных путей \cite{github_pgrouting}. Сервис реализован на языке Python с использованием фреймворка FastAPI, обеспечивающего высокую производительность за счет асинхронной обработки I/O операций.

\subsection{Предотвращение ложных топологических пересечений (Layer Separation)}

\paragraph{Проблема.}
Стандартный алгоритм построения топологии (\texttt{ST\_Node}) обрабатывает геометрии в двумерном пространстве. Если подать на вход все дорожные линии без предварительной обработки, система создаст узлы (перекрестки) в точках визуального пересечения дорог, проходящих на разных уровнях (например, мост над шоссе или тоннель под рекой). Это приводит к созданию некорректного маршрутного графа, где агент может совершить поворот с эстакады прямо на перпендикулярную улицу внизу.

\paragraph{Реализация.}
Для решения этой проблемы реализован конвейер предварительной обработки (см. рис. \ref{fig:build-flow}), включающий этап разделения слоев.
Перед построением топологии дороги жестко разделяются на две группы:
\begin{enumerate}
    \item \textbf{Ground Layer}: дороги, лежащие на земле (layer=0 или NULL).
    \item \textbf{Isolated Layer}: дороги, находящиеся на мостах (bridge=yes, layer>=1) или в тоннелях (tunnel=yes, layer<0).
\end{enumerate}

Алгоритм \texttt{pgr\_nodeNetwork} запускается независимо для каждой группы. Для изолированного слоя шаг поиска пересечений пропускается, либо выполняется только между объектами с одинаковым значением тега \texttt{layer}.
Это гарантирует корректность графа на многоуровневых развязках, таких как Третье транспортное кольцо (ТТК) в Москве.

\begin{figure}[H]
    \centering
    \includemermaid[width=\linewidth, height=0.45\textheight, keepaspectratio]{03-build-flow}
    \caption{Пайплайн построения топологии графа с разделением слоев}
    \label{fig:build-flow}
\end{figure}

\subsection{Многопоточная сборка графа и алгоритм Grid Partitioning}

\paragraph{Проблема.}
Построение единого связного графа на территорию Московской агломерации (более 2.5 млн ребер) является вычислительно сложной задачей. При последовательной обработке (Single-threaded) время построения топологии составляло более 40 минут. Кроме того, загрузка всего графа в оперативную память приводила к исчерпанию ресурсов (OOM Killer) на узлах с 8 ГБ RAM.

\paragraph{Реализация.}
Для ускорения процесса применен метод параллельной обработки с использованием пула процессов (\texttt{multiprocessing.Pool}). Алгоритм разбиения на сетку (Grid Partitioning), описанный во второй главе, позволяет обрабатывать отдельные тайлы независимо друг от друга \cite{dagstuhl_edge_hierarchies_2019}.

Процесс сборки реализован следующим образом:
\begin{enumerate}
    \item Глобальная область (Bounding Box) делится на $N \times M$ секторов.
    \item Главный процесс-координатор распределяет задачи по построению подграфов между воркерами.
    \item Каждый воркер загружает данные своего сектора из PostGIS, строит локальную топологию и возвращает список граничных узлов.
    \item Координатор «сшивает» (Merge) результаты, объединяя дублирующиеся узлы на границах секторов.
\end{enumerate}

\paragraph{Преимущество.}
Распараллеливание задачи на 8 ядер CPU позволило сократить время сборки графа с 40 до 13 минут. Потребление оперативной памяти снизилось с 12 ГБ (при монолитной сборке) до 370 МБ на один процесс, что сделало возможным развертывание системы на стандартном оборудовании.

\subsection{Оптимизация взаимодействия с базой данных}

\paragraph{Проблема.}
При высокой нагрузке (сотни запросов в секунду) затраты на установку TCP-соединения с базой данных становятся существенными. Стандартные синхронные драйверы (psycopg2) блокируют поток выполнения на время ожидания ответа от СУБД, что снижает пропускную способность сервиса (Throughput).

\paragraph{Реализация.}
В качестве драйвера базы данных выбран \texttt{asyncpg}, реализующий бинарный протокол PostgreSQL и полную поддержку цикла событий \texttt{asyncio}.
Реализован механизм постоянного пула соединений (Connection Pool). Пул инициализируется при старте приложения и поддерживает открытыми от 10 до 50 соединений.
Настройка пула:
\begin{code}{asyncpg-pool}{Python}{Инициализация пула соединений asyncpg}
pool = await asyncpg.create_pool(
    dsn=database_url,
    min_size=10,
    max_size=50,
    command_timeout=60,
    statement_cache_size=0  # Отключено для экономии памяти
)
\end{code}

\paragraph{Преимущество.}
Использование пула соединений устранило накладные расходы на рукопожатие (Handshake) при каждом запросе. Бенчмарки показали, что переход на \texttt{asyncpg} увеличил пропускную способность сервиса в 3.5 раза по сравнению с синхронной реализацией на SQLAlchemy ORM.

\subsection{Оптимизация сериализации геометрии (WKB vs WKT)}

\paragraph{Проблема.}
Передача геометрии маршрута от базы данных к сервису в текстовом формате WKT (Well-Known Text) создает избыточную нагрузку на канал связи и процессор, требуя парсинга длинных строк вида \texttt{"LINESTRING(37.61 55.75, ...)"}. Для сложных маршрутов объем текстовых данных достигал нескольких мегабайт.

\paragraph{Реализация.}
Взаимодействие переведено на бинарный формат WKB (Well-Known Binary). Функция PostGIS \texttt{ST\_AsBinary} возвращает компактное байтовое преставление геометрии, которое напрямую десериализуется в объекты Python \texttt{shapely.geometry} или передается клиенту без декодирования \cite{postgis_asmvtgeom, postgis_wms_utwente}.

Сравнение объемов данных для маршрута длиной 25 км:
\begin{itemize}
    \item \textbf{WKT (Text):} 485 КБ. Парсинг: 120 мс.
    \item \textbf{WKB (Binary):} 142 КБ. Парсинг: 15 мс.
\end{itemize}

\paragraph{Преимущество.}
Переход на бинарный формат сократил объем передаваемого трафика в 3.4 раза и ускорил обработку ответов базы данных на порядок. Это критически важно для соблюдения SLA по времени построения маршрута (< 200 мс на сам поиск, < 50 мс на передачу).

\subsection{Конвейер обработки запроса маршрутизации}

Жизненный цикл запроса на построение маршрута включает в себя несколько этапов валидации и предварительной обработки входных данных. На рисунке \ref{fig:runtime-flow} показана последовательность действий системы от получения REST-запроса до возврата GeoJSON/WKB ответа.

\begin{figure}[H]
    \centering
    \includemermaid[width=\linewidth, height=0.5\textheight, keepaspectratio]{03-runtime-flow}
    \caption{Конвейер обработки запроса маршрутизации}
    \label{fig:runtime-flow}
\end{figure}

Ключевые этапы конвейера:
\begin{enumerate}
    \item \textbf{Smart Snapping (KNN).} Входные координаты (широта/долгота) привязываются к ближайшему ребру дорожного графа с использованием индекса GiST (оператор \texttt{<->}) \cite{postgis_vision_2018}.
    \item \textbf{Dynamic BBOX.} Для оптимизации поиска вычисляется ограничивающий прямоугольник, охватывающий старт и финиш, с динамическим отступом $\delta$.
    \item \textbf{SQL Routing.} Выполняется функция \texttt{pgr\_dijkstra} внутри базы данных.
    \item \textbf{Geometry Merge.} Найденные сегменты объединяются в единую линию \texttt{ST\_MakeLine}, и, при необходимости, к маршруту добавляются "хвосты" от реальных координат пользователя до узлов графа (Zig-Zag Fix).
\end{enumerate}

\section{Реализация подсистемы межсервисного взаимодействия (Шлюз)}

Подсистема межсервисного взаимодействия (API Gateway) выполняет роль единой точки входа для всех внешних клиентов, обеспечивая маршрутизацию запросов, балансировку нагрузки и сокрытие внутренней топологии микросервисов.

\subsection{Унификация доступа и декларативная маршрутизация}

В микросервисной архитектуре клиентское приложение вынуждено знать адреса десятков внутренних сервисов (Data Processor, Router, Auth Service и т.д.). При изменении адресации или масштабировании (добавлении реплик) требуется обновление всех клиентов. Жесткое кодирование (hardcode) маршрутов в коде самого шлюза также приводит к высокой связности: добавление нового сервиса требует пересборки и перезапуска Gateway.

Для решения данной задачи внедрен паттерн «Реестр маршрутов» (Route Registry). Конфигурация всех проксируемых путей вынесена в внешний YAML-файл, который загружается при старте приложения.
Пример конфигурации:
\begin{code}{gateway-routes-yaml}{YAML}{Пример конфигурации реестра маршрутов}
routes:
  - path: /api/v1/map
    service: http://data-processor:8000
    strip_prefix: false
  - path: /api/v1/route
    service: http://router-service:8001
    methods: [POST]
\end{code}

При запуске Шлюз читает реестр и динамически генерирует обработчики (Handlers) для каждого маршрута.

Благодаря этому клиент взаимодействует только с одним хостом (\texttt{api.gateway.local}), не зная о внутренней сложности системы. Изменение топологии бэкенда производится правкой конфигурационного файла без остановки сервиса (при наличии механизма Hot Reload).

\subsection{Проблема захвата переменной цикла (Loop Variable Capture)}

В процессе реализации динамической регистрации маршрутов в цикле \texttt{for} была выявлена ошибка, связанная с особенностями работы замыканий в Python (Late Binding).
Рассмотрим наивную реализацию:
\begin{code}{naive-loop-error}{Python}{Ошибочная реализация регистрации обработчиков}
for route in routes:
    # ОШИБКА: route берется из внешней области видимости
    app.add_route(route.path, lambda req: proxy(req, route.url))
\end{code}

По завершении цикла переменная \texttt{route} принимает значение последнего элемента списка. В результате все зарегистрированные эндпоинты (например, \texttt{/map} и \texttt{/route}) начинали перенаправлять запросы на один и тот же последний сервис.

Для фиксации контекста выполнения применен паттерн «Фабрика замыканий» (Closure Factory). Создана отдельная функция высшего порядка, которая принимает текущее значение переменной как аргумент и возвращает новую асинхронную функцию-обработчик. Это создает изолированную область видимости (Closure Scope) для каждой итерации цикла.

\begin{code}{closure-factory}{Python}{Фабрика замыканий для динамических маршрутов}
def create_proxy_handler(target_url: str):
    # target_url "заморожен" в локальной области видимости этой функции
    async def handler(request: Request):
        return await forward_request(request, target_url)
    return handler

for route in registry:
    handler = create_proxy_handler(route.service_url)
    app.add_route(route.path, handler)
\end{code}

Применение фабрики замыканий гарантировало корректную маршрутизацию запросов. Этот подход является идиоматическим для Python и устраняет риск возникновения трудноуловимых багов при динамическом маппинге функций.

\subsection{Туннелирование WebSocket-соединений}

Протокол HTTP является stateless (без сохранения состояния), тогда как WebSocket требует поддержания постоянного TCP-соединения. Стандартные библиотеки проксирования (например, \texttt{httpx}) не умеют «пробрасывать» (Upgrade) HTTP-соединение до WebSocket и прозрачно передавать бинарный поток данных в обе стороны.

В рамках работы разработан механизм асинхронного туннелирования на базе библиотеки \texttt{websockets} и примитивов конкурентности \texttt{asyncio}.
При установке соединения Шлюз выступает в роли «человека посередине» (Man-in-the-Middle):
\begin{itemize}
    \item Принимает \texttt{Sec-WebSocket-Key} от клиента.
    \item Инициирует рукопожатие с целевым микросервисом.
    \item Запускает две независимые корутины для пересылки сообщений.
\end{itemize}

Для управления жизненным циклом соединения используется функция \texttt{asyncio.gather}, которая ожидает завершения любой из двух задач пересылки:
\begin{code}{asyncio-tunnel}{Python}{Асинхронное управление задачами туннелирования}
upstream_task = asyncio.create_task(proxy_stream(client_ws, upstream_ws))
downstream_task = asyncio.create_task(proxy_stream(upstream_ws, client_ws))
# Ожидание разрыва соединения с любой стороны
done, pending = await asyncio.wait(
    [upstream_task, downstream_task],
    return_when=asyncio.FIRST_COMPLETED
)
# Отмена оставшейся задачи
for task in pending: task.cancel()
\end{code}

\begin{figure}[H]
    \centering
    \includemermaid[width=\linewidth, height=0.6\textheight, keepaspectratio]{03-gateway-seq}
    \caption{Диаграмма последовательности проксирования WebSocket}
    \label{fig:gateway-seq}
\end{figure}

На рисунке \ref{fig:gateway-seq} показана последовательность обмена сообщениями. Шлюз прозрачно транслирует коды закрытия (Close Codes) и бинарные фреймы, не вмешиваясь в содержимое пакетов.

Реализованный туннель позволяет строить событийно-ориентированные системы (Event-Driven), где сервер может инициировать отправку данных (Push Notifications) клиенту. Использование асинхронного I/O позволяет одному экземпляру шлюза обслуживать тысячи активных WebSocket-сессий с потреблением памяти менее 10 КБ на соединение.

\section{Разработка клиентского модуля и механизмов интеграции}

Клиентская часть комплекса (Qt Client) реализована как гибридное настольное приложение, объединяющее нативный функционал и современные веб-технологии визуализации.

\subsection{Пользовательский интерфейс и управление}

Главное окно приложения (рисунок~\ref{fig:client-main}) разделено на две функциональные области: интерактивную карту и боковую панель управления (Sidebar). Боковая панель содержит набор виджетов для управления процессом маршрутизации:

\begin{itemize}
    \item \textbf{Панель маршрутных точек (Points Panel):} Позволяет пользователю задавать начальную (Start), конечную (End) и промежуточные (Via) точки маршрута. Реализован функционал добавления, удаления и изменения порядка следования точек. Координаты могут быть введены вручную или выбраны кликом по карте.
    \item \textbf{Панель результатов (Routes Panel):} Отображает список найденных альтернативных маршрутов (K-Shortest Paths). Для каждого маршрута выводятся ключевые метрики: общая протяженность (км), расчетное время в пути (мин) и количество ребер графа. Выбор элемента в списке подсвечивает соответствующую траекторию на карте.
    \item \textbf{Управление симуляцией (Simulation Panel):} Блок элементов интерфейса (<<Sim Speed>>, <<FPS>>, управление агентом), зарезервированный для подсистемы мультиагентного моделирования. В текущей версии системы эти элементы демонстрируют готовность архитектуры к внедрению динамических агентов (автомобилей), движущихся по построенным маршрутам, но их функционал пока ограничен базовой отладкой.
\end{itemize}

\begin{figure}[H]
    \centering
    \includegraphics[width=0.95\linewidth]{gui-routes}
    \caption{Главное окно приложения. Слева -- панель управления с параметрами маршрута (K=5) и списком найденных путей. На карте отображены 5 маршрутов (синим цветом выбранный в панели результатов маршрут, серым -- остальные), проходящих через заданные точки.}
    \label{fig:client-main}
\end{figure}

\subsection{Гибридная архитектура и выбор движка рендеринга}

Классические виджеты Qt (QGraphicsView) не предназначены для отрисовки миллионов векторных объектов с частотой 60 кадров в секунду. Попытка реализации карты на чистом C++/Python привела бы к необходимости написания собственной системы тайлинга и управления GPU-шейдерами. С другой стороны, чистое веб-приложение в браузере ограничено <<песочницей>> (Sandbox) и не имеет прямого доступа к файловой системе и оборудованию, что критически важно для целей исследования (сбор локальных метрик, работа без интернета).

В качестве архитектурного решения выбрана концепция <<толстого клиента>> с использованием компонента \texttt{QWebEngineView} (на базе Chromium). Рендеринг карты делегирован библиотеке \textbf{MapLibre GL JS}, использующей WebGL для аппаратного ускорения графики \cite{maplibre_main, jawg_maplibre_vs_leaflet, maplibre_vectortile_source, maplibre_hybrid_terrain}.
Основная логика приложения (меню, управление симуляцией, сетевой обмен) реализована на Python (PyQt5), а визуализация -- на JavaScript.

В архитектуре системы реализован двухступенчатый конвейер преобразования координат для генерации тайлов (MVT).

\textbf{Этап 1: Проецирование (EPSG:4326 $\to$ EPSG:3857).}
Исходные данные OpenStreetMap поступают в географической системе координат WGS 84 (широта/долгота). Поскольку стандарт векторных тайлов Mapbox Vector Tile требует предварительного проецирования в плоскую метрическую систему Web Mercator, данное преобразование выполняется на этапе импорта данных (см. \texttt{queries.py}). В базе данных хранятся обе геометрии: оригинальная (\texttt{geom}) и спроецированная (\texttt{geom\_3857}). Это позволяет избежать ресурсоемких вычислений \texttt{ST\_Transform} при каждом запросе тайла.

Формула прямого преобразования Меркатора:
\begin{equation}
\begin{cases}
x = R \cdot \lambda \\
y = R \cdot \ln\left(\tan\left(\frac{\pi}{4} + \frac{\phi}{2}\right)\right)
\end{cases},
\end{equation}
\where{
    $R$ & радиус Земли (6378137 м); \\
    $\lambda$ & долгота в радианах; \\
    $\phi$ & широта в радианах.
}

\textbf{Этап 2: Аффинное преобразование в координаты тайла.}
Для передачи на клиент спроецированные координаты (метры) переводятся в локальную целочисленную систему координат тайла (Screen Coordinates) с помощью матрицы масштабирования и сдвига:
\begin{equation}
\begin{pmatrix} X_{tile} \\ Y_{tile} \end{pmatrix} = \begin{pmatrix} S/W_z & 0 \\ 0 & -S/W_z \end{pmatrix} \begin{pmatrix} x_{merc} - x_{origin} \\ y_{merc} - y_{origin} \end{pmatrix},
\end{equation}
\where{
    $X_{tile}, Y_{tile}$ & координаты точки внутри тайловой системы координат (экранные координаты, обычно целые числа); \\
    $S$ & разрешение тайла (Extent = 4096 единиц); \\
    $W_z$ & ширина области, охватываемой тайлом на уровне масштаба $z$ (в метрах); \\
    $x_{merc}, y_{merc}$ & координаты исходной точки в глобальной проекции Web Mercator (EPSG:3857); \\
    $x_{origin}, y_{origin}$ & координаты левого верхнего угла тайла в проекции Web Mercator.
}

Данный подход обеспечивает баланс между точностью и скоростью: хранение в EPSG:3857 ускоряет выборку (индекс GiST работает по плоским координатам), а локальные координаты тайла снимают проблему дрожания (Jittering) на клиенте из-за ограничений Float32 в GPU.

Гибридный подход объединил производительность PyQt5 и гибкость веб-разработки. Использование MVT перенесло вычислительную нагрузку по проекции (Projection) и упрощению геометрии (Simplification) на сервер БД\cite{crunchy_dynamic_tiles}, освободив ресурсы клиента для плавной анимации интерфейса.

\subsection{Механизм мостов QWebChannel}

Существенным препятствием при интеграции является изоляция процессов Python и JavaScript, которая препятствует прямому вызову функций. Использование HTTP-сервера только для межпроцессного взаимодействия (IPC) внутри одного приложения добавляет неоправданную задержку (Latency ок. 10-15 мс) и усложняет архитектуру.

Для организации двунаправленного канала связи применен протокол \texttt{QWebChannel} поверх WebSocket транспортного уровня. Разработаны специализированные объекты-мосты (Bridges), которые экспортируются из контекста Python в глобальную область видимости \texttt{window} браузера:

\begin{itemize}
    \item \textbf{ZoomBridge.} Отслеживает уровень масштаба карты. При изменении зума (событие \texttt{zoomend} в JS) он уведомляет бэкенд о необходимости пересчета уровня детализации (LOD).
    \item \textbf{ConfigBridge.} Передает JSON-схему стилей (Map Style Spec)\cite{maplibre_style_spec} при инициализации карты, позволяя управлять цветами дорог и толщиной линий из единого конфигурационного файла приложения.
    \item \textbf{LoggerBridge.} Перехватывает вызовы \texttt{console.log / error} в JavaScript и транслирует их в стандартный поток вывода Python. Это позволяет видеть ошибки JS-кода в общем логе systemd.
\end{itemize}

\begin{figure}[H]
    \centering
    \includemermaid[width=\linewidth, height=0.85\textheight, keepaspectratio]{03-qwebchannel-seq}
    \caption{Схема взаимодействия через QWebChannel}
    \label{fig:qwebchannel-seq}
\end{figure}

На диаграмме \ref{fig:qwebchannel-seq} представлены четыре основных сценария взаимодействия сред выполнения.
\begin{enumerate}
    \item \textbf{Инициализация}. При старте приложения Python-бэкенд регистрирует объекты-мосты (\texttt{config}, \texttt{logger}, \texttt{zoom}). После загрузки страницы во фронтенд инъектируется объект \texttt{window.qt}, обеспечивающий доступ к методам Python из JavaScript.
    \item \textbf{Сценарий 1: Перехват логов}. Для централизованной диагностики переопределен метод \texttt{console.error}. Сообщения об ошибках тайлов или скриптов транслируются через \texttt{LoggerBridge} в стандартный поток ошибок Python (stderr), что позволяет агрегировать логи фронтенда и бэкенда в единой системе логгирования \texttt{loguru}.
    \item \textbf{Сценарий 2: Синхронизация LOD}. При изменении масштаба карты (событие \texttt{zoomend}) JavaScript вызывает метод \texttt{zoom.onZoomChanged}. Python-бэкенд получает новое значение $Z$, пересчитывает стратегию детализации и обновляет параметры SQL-фильтрации через Data Processor.
    \item \textbf{Сценарий 3: Управление камерой}. Обратный канал связи реализуется через метод \texttt{runJavaScript}. Например, при перемещении мышью по карте Python генерирует команду \texttt{map.flyTo({...})}, инициируя плавную WebGL-анимацию полета камеры в выбранном направлении.
\end{enumerate}

\subsection{Реактивная синхронизация детализации (LOD)}

При отображении карты на мелком масштабе (например, весь город), попытка загрузить все дороги (включая дворовые проезды) приводит к перегрузке видеопамяти и визуальному шуму. Необходима динамическая фильтрация контента в зависимости от зума (Zoom Level).

Для оптимизации рендеринга реализован следующий реактивный цикл синхронизации:
\begin{enumerate}
    \item пользователь меняет масштаб карты колесом мыши;
    \item модуль JS вызывает метод ZoomBridge $\rightarrow$ модуля Python;
    \item модуль Python отправляет новый уровень LOD (Level of Detail) в сервис Data Processor через WebSocket;
    \item сервис Data Processor обновляет параметры SQL-фильтрации (например, \texttt{WHERE highway IN ('motorway', 'primary')});
    \item карта перезагружает тайлы.
\end{enumerate}

Полное время цикла синхронизации (Round Trip Time) описывается формулой:
\begin{equation}
T_{sync} \approx T_{ws\_latency} + T_{config\_parse} + T_{invalidate\_cache} \approx 65 \text{ мс},
\end{equation}
\where{
    $T_{ws\_latency}$ & сетевая задержка WebSocket; \\
    $T_{config\_parse}$ & время парсинга конфигурации; \\
    $T_{invalidate\_cache}$ & время инвалидации кэша тайлов.
}

Такая задержка является незаметной для человеческого глаза (порог восприятия 100 мс).

Динамический LOD снизил нагрузку на рендеринг в 10 раз при просмотре обзорных карт, скрывая до 90~\% мелких объектов. Это позволило добиться стабильных 60 FPS даже на интегрированных видеокартах.

Эффективность работы алгоритма динамической детализации наглядно представлена на рисунке \ref{fig:client-lod}. На низких уровнях масштабирования (z9) рендеринг ограничивается только магистральной сетью, что обеспечивает чистоту восприятия. По мере приближения (z11--z15) система последовательно подгружает локальные дороги, дворовые проезды и служебные пути.

\begin{figure}[H]
    \centering
    % --- Первая строка ---
    \begin{subfigure}[t]{0.48\linewidth}
        \centering
        \includegraphics[width=\linewidth]{gui-lod-1}
        \caption{Низкий масштаб (z9): только магистрали}
        \label{fig:lod-z9}
    \end{subfigure}
    \hfill
    \begin{subfigure}[t]{0.48\linewidth}
        \centering
        \includegraphics[width=\linewidth]{gui-lod-2}
        \caption{Средний масштаб (z11): региональные дороги}
        \label{fig:lod-z11}
    \end{subfigure}
    
    \vspace{1em}
    % --- Вторая строка ---
    \begin{subfigure}[t]{0.48\linewidth}
        \centering
        \includegraphics[width=\linewidth]{gui-lod-3}
        \caption{Высокий масштаб (z13): уличная сеть}
        \label{fig:lod-z13}
    \end{subfigure}
    \hfill
    \begin{subfigure}[t]{0.48\linewidth}
        \centering
        \includegraphics[width=\linewidth]{gui-lod-4}
        \caption{Макс. детализация (z15): дворы}
        \label{fig:lod-z15}
    \end{subfigure}
    \caption{Демонстрация работы системы LOD: прогрессивная детализация дорожной сети}
    \label{fig:client-lod}
\end{figure}

\subsection{Визуальное кодирование и атрибутивный состав}

Для улучшения читаемости карты оператором реализована система цветового кодирования классов дорог (см. таблицу \ref{tab:road-attributes}). Каждому функциональному классу (тэг `highway`) сопоставлен уникальный цвет и толщина линии, что позволяет мгновенно оценивать связность дорожной сети.
Кроме того, в системе зафиксированы расчетные скоростные лимиты для каждого класса, используемые при построении изохрон и маршрутов.

\begin{table}[H]
    \centering
    \caption{Цветовое кодирование типов дорог и скоростной режим}
    \label{tab:road-attributes}
    \small
    \begin{tabular}{|l|l|c|}
        \hline
        \textbf{Тип дороги (OSM tag)} & \textbf{Цвет (HEX)} & \textbf{Speed (км/ч)} \\ \hline
        Motorway & \textcolor[HTML]{1E40AF}{\rule{1cm}{3mm}} \#1E40AF (Dark Blue) & 110 \\ \hline
        Trunk & \textcolor[HTML]{6200FF}{\rule{1cm}{3mm}} \#6200FF (Purple) & 110 \\ \hline
        Primary & \textcolor[HTML]{9C00AA}{\rule{1cm}{3mm}} \#9C00AA (Magenta) & 60 \\ \hline
        Secondary & \textcolor[HTML]{FF5EFF}{\rule{1cm}{3mm}} \#FF5EFF (Pink) & 60 \\ \hline
        Tertiary & \textcolor[HTML]{FF2E2E}{\rule{1cm}{3mm}} \#FF2E2E (Red) & 60 \\ \hline
        Unclassified & \textcolor[HTML]{5C5C5C}{\rule{1cm}{3mm}} \#5C5C5C (Gray) & 60 \\ \hline
        Residential / Living & \textcolor[HTML]{FF8635}{\rule{1cm}{3mm}} \#FF8635 (Orange) & 20 \\ \hline
        Service & \textcolor[HTML]{008D0C}{\rule{1cm}{3mm}} \#008D0C (Green) & 20 \\ \hline
    \end{tabular}
\end{table}

\section{Инфраструктурные компоненты и среда развертывания}

Для обеспечения надежности, переносимости и наблюдаемости системы был разработан комплекс инфраструктурных компонентов, отвечающих за конфигурацию, логирование и оркестрацию контейнеров.

\subsection{Иерархическая система конфигурации}

\paragraph{Проблема.}
В распределенных системах управление конфигурацией часто становится источником ошибок. Хранение настроек в коде (Hardcode) делает систему негибкой. Использование простых JSON-файлов не позволяет переиспользовать общие блоки (например, настройки подключения к БД) между различными сервисами, что приводит к дублированию. Кроме того, необходимо обеспечить возможность переопределения параметров при развертывании в контейнерах (через переменные окружения) без правки файлов.

\paragraph{Реализация.}
Разработана гибридная система конфигурации на базе библиотеки \textbf{Pydantic}. Она объединяет строго типизированные модели данных Python с гибкостью YAML-файлов.
Для поддержки модульности реализован кастомный загрузчик YAML, поддерживающий директиву \texttt{!include}. Это позволяет разбить монолитный конфиг на логические части:
\begin{code}{yaml-include}{YAML}{Использование директивы include в YAML}
database: !include common/database.yaml
logging: !include common/logging.yaml
server:
  host: 0.0.0.0
  port: 8000
\end{code}

Поверх файловой конфигурации работает механизм переопределения через переменные окружения (Environment Variables). Используется префикс \texttt{APP\_\_} и двойное подчеркивание как разделитель вложенности.
Например, переменная \texttt{APP\_\_DATABASE\_\_HOST=10.0.0.5} автоматически переопределит значение \texttt{database.host} в объекте настроек.

\paragraph{Преимущество.}
Типизация Pydantic обеспечивает валидацию настроек на старте приложения (Fail Fast): сервис не запустится, если вместо числа в порт передан текст. Механизм \texttt{!include} устранил дублирование конфигурации между четырьмя микросервисами. Переменные окружения позволили инъектировать секреты (пароли БД) и адреса сервисов в Docker-контейнеры, не сохраняя их в репозитории кода.

\subsection{Отказоустойчивое логирование (Zero-Effect Logging)}

\paragraph{Проблема.}
В высоконагруженных асинхронных приложениях операции ввода-вывода (I/O) при записи логов могут заблокировать цикл событий (Event Loop). Если диск перегружен или сетевой лог-колектор недоступен, вызов \texttt{print()} или стандартного \texttt{logging} может остановить обработку всех запросов на сотни миллисекунд.

\paragraph{Реализация.}
Внедрена библиотека \textbf{loguru}, настроенная в режиме асинхронного стока (Asynchronous Sink).
Параметр \texttt{enqueue=True} гарантирует, что запись сообщения в файл или отправка по сети выполняется в отдельном потоке (Thread), не блокируя основной цикл приложения:
\begin{code}{loguru-setup}{Python}{Настройка асинхронного стока логов}
logger.add(
    "logs/app.json",
    rotation="500 MB",
    serialize=True,  # JSON формат
    enqueue=True,    # Асинхронная запись
    compression="zip"
)
\end{code}

Также используется принцип «ленивой оценки» (Lazy Evaluation) аргументов логирования. Форматирование строки \texttt{logger.debug("Data: \{data\}", data=heavy\_object)} происходит только если уровень DEBUG включен. В продакшн-режиме (INFO) тяжелая сериализация объекта не выполняется, экономя ресурсы CPU.

\paragraph{Преимущество.}
Концепция «Zero-Effect Logging» минимизировала влияние подсистемы наблюдения на производительность. Время ответа (Latency) API практически не зависит от объема генерируемых логов. Структурированный JSON-вывод позволил автоматизировать сбор метрик в ELK Stack (Elasticsearch, Logstash, Kibana).

\subsection{Контейнеризация и оркестрация}

\paragraph{Проблема.}
Развертывание сложного комплекса из 5 контейнеров (Gateway, Data Processor, Router, PostGIS, Frontend) вручную чревато ошибками конфигурации сети ("Dependency Hell") и конфликтами портов. Требуется гарантия идентичности окружения разработки и эксплуатации.

\paragraph{Реализация.}
Среда развертывания описана декларативно в манифесте \texttt{docker-compose.yml}.
Реализована строгая изоляция сетевых контуров:
\begin{itemize}
    \item \textbf{backend-net}: Внутренняя сеть для межсервисного взаимодействия. Недоступна извне.
    \item \textbf{frontend-net}: Публичная сеть, в которую смотрит только API Gateway (порт 80/443).
\end{itemize}

Для обеспечения целостности запуска (Startup Order) настроены проверки жизнеспособности (Health-checks).
Сервис \texttt{router-service} имеет зависимость \texttt{service\_healthy} от базы данных:
\begin{code}{docker-healthcheck}{YAML}{Конфигурация проверки жизнеспособности зависимостей}
router:
  depends_on:
    db:
      condition: service_healthy
db:
  healthcheck:
    test: ["CMD-SHELL", "pg_isready -U postgres"]
    interval: 5s
    timeout: 5s
    retries: 5
\end{code}

\paragraph{Преимущество.}
Контейнеризация обеспечила полную воспроизводимость развертывания (Infrastructure as Code). Изоляция сетей повысила безопасность, исключив прямой доступ к БД из интернета. Health-checks предотвращают "падение" зависимых сервисов при старте, автоматически перезапуская их до готовности инфраструктуры.


\section{Выводы по главе}

Разработанная программная реализация системы демонстрирует применение современных архитектурных паттернов для решения задач высокопроизводительной обработки геоданных.
Внедрение асинхронного шлюза с Websocket-туннелированием обеспечило возможность работы в реальном времени.
Специализированные алгоритмы генерации тайлов (Soft BBox, Z-order) позволили достичь высокого качества визуализации.
Инфраструктурные решения (Zero-Effect Logging, Config setup) гарантируют эксплуатационную надежность и наблюдаемость комплекса.
