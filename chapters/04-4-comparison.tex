\section{Сравнительная характеристика решений}

Для позиционирования разработанной системы в ряду существующих аналогов было проведено прямое сравнение с эталонным Open Source решением — **OSRM (Open Source Routing Machine)**.
Сравнение проводилось на одном и том же наборе данных (Москва) и идентичном оборудовании.

\subsection{Производительность vs Гибкость}

Результаты измерений (таблица~\ref{tab:comparison-osrm}) демонстрируют фундаментальное различие архитектурных подходов: In-Memory (CH) против Database-Centric (SQL).

\begin{table}[H]
    \centering
    \caption{Сравнительная характеристика разработанной системы и OSRM}
    \label{tab:comparison-osrm}
    \begin{tabularx}{\linewidth}{|p{4.5cm}|X|X|}        
        \hline
        \textbf{Характеристика} & \textbf{Разработанная система (PostgreSQL)} & \textbf{OSRM (Contracted)} \\ \hline
        Latency (Median) & $\approx$ 6000 мс (Cold) / 2500 мс (Warm) & $\approx$ 10 мс \\ \hline
        RAM Usage & 370 МБ (Shared Buffers) & 4-8 ГБ (In-Memory Graph) \\ \hline
        Graph Update Time & $\mathbf{0 \text{ мс} (Real-time)}$ & 30-60 мин (Rebuild CH) \\ \hline
        Dynamic Weights & \textbf{Да (SQL Function)} & Нет (Static Profile) \\ \hline
    \end{tabularx}
\end{table}

Хотя OSRM выигрывает в скорости обработки запроса на три порядка (10 мс против 6 с), он требует полной перестройки графа (Contraction Hierarchies) при любом изменении весов ребер\cite{lazarsfeld_ch_dijkstra}. В условиях задачи мультиагентной навигации, где веса меняются динамически (пробки, аварии, перекрытия), задержка в 30-60 минут является неприемлемой \cite{geisberger2008_semantic, geisberger2008_scispace, geisberger2008_researchgate, dblp_geisberger}.
Наша система позволяет менять весовой коэффициент любого ребра мгновенно (SQL \texttt{UPDATE}), и следующий же запрос маршрута учтет это изменение.

\subsection{Итоговый вывод}

Разработанная архитектура является оптимальным выбором для систем \textbf{логистики реального времени} (Soft Real-Time Logistics), где критически важна возможность учета внезапных изменений графа (динамические штрафы), и допустима задержка отклика в пределах нескольких секунд.
Для приложений класса "Навигатор в авто", требующих перестроения маршрута за 100 мс, предпочтительнее использовать гибридные подходы (например, Valhalla), однако они значительно сложнее в реализации и развертывании \cite{arxiv_parallel_ch_2024}.
Потребление памяти (370 МБ) позволяет разворачивать систему даже на встраиваемых устройствах (Edge Computing), что недостижимо для классических CH-решений.
