\section{Сравнительная характеристика решений}

Для позиционирования разработанной системы относительно существующих аналогов проведено прямое сравнение с эталонным движком с открытым исходным кодом OSRM. Сравнение характеристик для графа масштаба Москвы приведено в таблице \ref{tab:comparison-osrm}.

\begin{table}[H]
    \centering
    \caption{Сравнительный анализ архитектурных подходов}
    \label{tab:comparison-osrm}
    \small
    \begin{tabularx}{\linewidth}{|p{5cm}|X|X|}
        \hline
        \textbf{Характеристика} & \textbf{Разработанная система (PostgreSQL)} & \textbf{OSRM (C++)} \\ \hline
        Архитектура & На базе СУБД & Предрасчитанный граф \\ \hline        
        Задержка отклика (поиск) & $\approx 6$~с (Последовательно) / $\approx 1.2$~с (Параллельно) & $10-25$~мс \cite{neis2012routing} \\ \hline
        Потребление ОЗУ (Москва) & $370$~МБ & $500$~МБ \cite{osrm2018memory} \\ \hline
        Время обновления графа & В реальном времени & $5-15$~мин \cite{malagis2025osrm} \\ \hline
        Учет динамических весов & Нативная поддержка & Не поддерживается \\ \hline
    \end{tabularx}
\end{table}

Как видно из таблицы, OSRM выигрывает в скорости поиска маршрута на два порядка за счет использования предварительно рассчитанных индексов и хранения всего графа в оперативной памяти. Однако этот подход имеет критический недостаток -- статичность данных. Любое изменение весового коэффициента, например, перекрытие дороги из-за аварии, требует полной перестройки индекса, что занимает до 15 минут \cite{malagis2025osrm} на масштабах рассматриваемого графа Москвы.

Разработанная система, уступая в <<чистой>> скорости чтения, обеспечивает ключевое преимущество -- актуализацию топологии в режиме реального времени. Изменение весов выполняется одной транзакцией SQL \texttt{UPDATE} и мгновенно учитывается при построении следующего маршрута. 

Это делает предложенную архитектуру оптимальным выбором для систем логистики реального времени, где критически важна адаптивность к дорожной ситуации, а задержка отклика в пределах нескольких секунд является допустимой платой за актуальность данных.