\chapter*{ВВЕДЕНИЕ}
\addcontentsline{toc}{chapter}{ВВЕДЕНИЕ}

Актуальность темы исследования обусловлена необходимостью создания независимых отечественных геоинформационных систем, способных работать в условиях высокой динамики дорожной обстановки. Существующие зарубежные решения часто являются закрытыми <<черными ящиками>> или требуют значительных вычислительных ресурсов для перестройки графа при изменении весов ребер.

Целью данной работы является разработка и реализация масштабируемой системы построения маршрутов на основе открытых данных OpenStreetMap (OSM) для Московской агломерации.

Для достижения поставленной цели необходимо решить следующие задачи:
\begin{enumerate}
    \item Провести анализ существующих технологий маршрутизации и выбрать оптимальный технологический стек.
    \item Исследовать методы получения и обработки больших объемов геоданных из OpenStreetMap.
    \item Разработать алгоритм построения дорожного графа, устойчивый к ограничениям по оперативной памяти.
    \item Реализовать алгоритмы поиска оптимальных и альтернативных маршрутов с возможностью динамического учета ограничений.
    \item Провести нагрузочное тестирование системы и оценить ее производительность.
\end{enumerate}

Объектом исследования являются алгоритмы поиска кратчайшего пути на графах большой размерности.
Предметом исследования является программная реализация системы маршрутизации на базе СУБД PostgreSQL/PostGIS.

Практическая значимость работы заключается в создании инструмента, позволяющего строить маршруты с учетом оперативной дорожной обстановки без необходимости полной перестройки графа, что критически важно для логистических и диспетчерских задач.
