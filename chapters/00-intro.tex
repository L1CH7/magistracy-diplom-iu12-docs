\chapter*{ВВЕДЕНИЕ}
\addcontentsline{toc}{chapter}{ВВЕДЕНИЕ}

\textbf{Актуальность темы.} Развитие автономного транспорта и мультиагентных логистических систем предъявляет качественно новые требования к алгоритмам построения маршрутов. Традиционные навигационные решения, такие как OSRM или GraphHopper, ориентированы на достижение минимальной задержки запроса за счет использования статических структур данных (иерархии сокращений — Contraction Hierarchies). Однако в условиях мегаполиса, характеризующегося высокой динамикой дорожной ситуации (аварии, ремонтные работы, оперативное перекрытие зон), такие системы становятся неэффективными: время перестройки статических индексов для графа масштаба московской агломерации составляет от 30 до 60 минут. Любое локальное изменение графа требует полной ревалидации данных, что создает критическую задержку в управлении транспортными потоками. В этой связи актуальной задачей является разработка навигационных комплексов на базе архитектуры \textit{Database-Centric}, обеспечивающих мгновенное обновление весов ребер графа (мгновенная актуализация дорожной ситуации) при сохранении высокой точности и вариативности маршрутов.

\textbf{Объектом исследования} являются алгоритмы и программные средства обеспечения навигации в динамически изменяющихся графовых структурах.

\textbf{Предметом исследования} является архитектура и программная реализация распределенной навигационной системы для мультиагентных транспортных комплексов.

\textbf{Цель работы} — разработка и исследование модульного программного комплекса для поиска и оптимизации маршрутов в динамической среде с использованием реляционной модели хранения графа.

Для достижения поставленной цели в работе решены следующие \textbf{задачи}:
\begin{enumerate}
    \item Проведен сравнительный анализ существующих систем маршрутизации и обоснована необходимость перехода к Database-Centric архитектуре для исключения этапа длительной предобработки данных.
    \item Разработан математический аппарат пространственного секционирования (Grid Partitioning), решивший проблему исчерпания памяти при обработке сверхбольших графов дорожных сетей.
    \item Спроектирована и реализована микросервисная архитектура системы на базе FastAPI и асинхронного взаимодействия с СУБД PostgreSQL/pgRouting.
    \item Разработан кроссплатформенный клиентский интерфейс на базе фреймворка Qt с применением гибридного рендеринга векторных карт (MapLibre GL).
    \item Проведено комплексное экспериментальное исследование характеристик системы на реальных данных Московской агломерации, включая анализ производительности, масштабируемости и качества семантического разнообразия маршрутов.
\end{enumerate}

\textbf{Методы исследования.} В работе применены методы теории графов, системного проектирования микросервисов, принципы обработки пространственных данных (PostGIS), а также методы нагрузочного тестирования и статистического анализа распределений.

\textbf{Научная новизна} работы заключается в реализации гибридного алгоритмического обеспечения, позволяющего сочетать транзакционную надежность реляционных баз данных с динамическим ограничением области поиска («Dynamic Bounding Box»), что позволило достичь времени отклика порядка 2.5 с без использования предварительно рассчитанных статических иерархий.

\textbf{Практическая значимость} заключается в создании готового программного решения, способного интегрироваться в системы управления БПЛА и флотилиями автономных доставщиков (soft real-time), обеспечивая адаптивную маршрутизацию в условиях постоянно меняющихся ограничений городской инфраструктуры.

\textbf{Положения, выносимые на защиту:}
\begin{enumerate}
    \item Методология построения графовой топологии через пространственное секционирование, позволившая снизить потребление памяти с 12.4 ГБ до 370 МБ.
    \item Событийно-ориентированный протокол «ленивой» загрузки данных (Lazy Loading) и динамической генерации векторных тайлов (On-Demand MVT).
    \item Алгоритмическая реализация поиска альтернативных путей на базе метода итеративных штрафов, обеспечивающая пространственное разнообразие маршрутов (индекс Жаккарда менее 3\%).
\end{enumerate}

\textbf{Структура и объем работы.} Дипломная работа состоит из введения, четырех глав, заключения и списка источников.

Во \textbf{введении} обоснована актуальность, определены цели, задачи и научная новизна исследования.

В \textbf{первой главе} выполнен анализ существующих решений и топологий, обоснован выбор технологического стека.

Во \textbf{второй главе} описаны математические основы системы: модель графа, секционирование и алгоритмы поиска разнородных альтернативных путей.

В \textbf{третьей главе} представлена программная реализация микросервисов (Router, Data Processor), шлюза API и клиентского приложения.

В \textbf{четвертой главе} приведены результаты испытаний, подтверждающие высокую производительность и логическую корректность системы в условиях реального мегаполиса.

В \textbf{заключении} сформулированы основные выводы по результатам выполненного исследования.
