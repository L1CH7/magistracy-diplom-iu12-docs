\chapter*{ВВЕДЕНИЕ}
\addcontentsline{toc}{chapter}{ВВЕДЕНИЕ}

\textbf{Актуальность темы} заключается в том, что развитие автономного транспорта и мультиагентных логистических систем предъявляет качественно новые требования к алгоритмам построения маршрутов. Традиционные навигационные решения, такие как OSRM или GraphHopper, ориентированы на достижение минимальной задержки запроса за счет использования статических структур данных (иерархии сокращений -- Contraction Hierarchies). Однако в условиях мегаполиса, характеризующегося высокой динамикой дорожной ситуации (аварии, ремонтные работы, оперативное перекрытие зон), такие системы становятся неэффективными: время перестройки статических индексов для графа масштаба московской агломерации составляет от 30 до 60 минут. Любое локальное изменение графа требует полной ревалидации данных, что создает критическую задержку в управлении транспортными потоками. В этой связи актуальной задачей является разработка навигационных комплексов на базе архитектуры с центрированием на базе данных, обеспечивающих мгновенное обновление весов ребер графа (мгновенная актуализация дорожной ситуации) при сохранении высокой точности и вариативности маршрутов.

\textbf{Объектом исследования} являются алгоритмы и программные средства обеспечения навигации в динамически изменяющихся графовых структурах.

\textbf{Предметом исследования} является архитектура и программная реализация распределенной навигационной системы для мультиагентных транспортных комплексов.

\textbf{Цель работы} -- разработка и исследование модульного программного комплекса для поиска и оптимизации маршрутов в динамической среде с использованием реляционной модели хранения графа.

Для достижения поставленной цели необходимо решить следующие \textbf{задачи}:
\begin{enumerate}
    \item Проанализировать методы маршрутизации в динамических графах и обосновать выбор архитектуры Database-Centric для систем реального времени.
    \item Разработать математическое и алгоритмическое обеспечение пространственного секционирования графа, решающее проблему исчерпания памяти при обработке сверхбольших дорожных сетей.
    \item Спроектировать и реализовать микросервисную архитектуру программного комплекса, обеспечивающую асинхронное взаимодействие компонентов и горизонтальную масштабируемость.
    \item Разработать кроссплатформенное клиентское приложение с гибридным механизмом рендеринга векторных карт для визуализации маршрутов.
    \item Провести экспериментальное исследование производительности системы и качества маршрутизации на реальных данных Московской агломерации.
\end{enumerate}

\textbf{Методы исследования.} При решении поставленных задач использовались методы теории графов (алгоритмы поиска кратчайших путей), методы системного анализа и проектирования программного обеспечения, принципы организации реляционных баз данных и геоинформационных систем (ГИС), а также методы статистического анализа данных.

\textbf{Научная новизна} работы заключается в разработке гибридного алгоритмического обеспечения на базе метода пространственного секционирования (Grid Partitioning), позволяющего эффективно обрабатывать сверхбольшие дорожные сети на ограниченных ресурсах (снижение потребления RAM с 12,4~ГБ до 370~МБ). В отличие от статических систем, предложенная архитектура обеспечивает принципиальную возможность мгновенной актуализации весов ребер графа без пересчета вспомогательных структур, что создает базу для реализации динамических навигационных алгоритмов.

\textbf{Практическая значимость} работы состоит в создании масштабируемой программной платформы для формирования маршрутов городского транспорта. Разработанная инфраструктура является технологическим фундаментом для развертывания мультиагентных систем управления: благодаря поддержке динамического обновления весов через транзакционные механизмы СУБД, система готова к интеграции модулей мониторинга пробок и оперативного диспетчерского управления без изменения архитектурного ядра.
