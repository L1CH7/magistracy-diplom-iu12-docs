\chapter{Исследование предметной области и выбор инструментов}
\label{ch:research}

\section{Анализ источников геоданных}

В качестве основного источника картографической информации выбран проект OpenStreetMap (OSM). В отличие от коммерческих аналогов (Google Maps, Яндекс.Карты), OSM предоставляет свободный доступ к сырым векторным данным под лицензией ODbL, что является критическим требованием для построения автономной системы маршрутизации.

\subsection{Архитектура данных OSM}
Схема хранения данных в проекте реализована в PostgreSQL и адаптирована под специфику навигационных задач. ER-диаграмма (рисунок~\ref{fig:osm-schema}) демонстрирует структуру таблиц, используемых для импорта.

\begin{figure}[H]
    \centering
    \includemermaid[width=0.55\linewidth]{02-osm-schema}
    \caption{Схема данных модуля загрузки (OSM Schema)}
    \label{fig:osm-schema}
\end{figure}

Ключевые сущности:
\begin{itemize}
    \item \textbf{osm.ways:} Хранит геометрию дорог. Поле `tags` (JSONB) обеспечивает гибкость добавления новых атрибутов без изменения схемы БД.
    \item \textbf{osm.cached\_tiles:} Журнал состояния загрузки тайлов. Позволяет координировать работу параллельных загрузчиков и исключает дублирование запросов.
\end{itemize}

\section{Методы извлечения данных (Overpass API)}

Для получения данных используется Overpass API — специализированный сервис для выполнения запросов к базе OSM. В отличие от скачивания полных дампов планеты (Planet.osm), этот подход позволяет загружать данные динамически, небольшими фрагментами (тайлами) по мере необходимости.

\subsection{Конструкция запроса Overpass QL}
Разработан специализированный запрос, фильтрующий данные на стороне сервера Overpass для минимизации трафика.

\begin{code}{overpass-ql}{C}{Overpass QL запрос для загрузки дорожного графа}
[out:json][timeout:300];
(
  // 1. Дорожная сеть (Whitelist)
  way["highway"~"^(motorway|trunk|primary| secondary|tertiary|residential|unclassified|service)$"]
    ({south},{west},{north},{east});

  // 2. Ограничения маневров (Turn Restrictions)
  relation["type"="restriction"]({south},{west},{north},{east});

  // 3. Точечные препятствия (Barriers)
  node["barrier"~"^(gate|bollard|lift_gate|block)$"]
    ({south},{west},{north},{east});
);
out body;
>; // Рекурсивный спуск для получения геометрии узлов
out skel qt;
\end{code}

Основные компоненты фильтрации:
\begin{enumerate}
    \item \textbf{Highway Whitelist:} Регулярное выражение отбирает только автомобильные дороги, исключая пешеходные тропы, велодорожки и технологические пути (`construction`), что сокращает объем данных на 60\%.
    \item \textbf{Turn Restrictions:} Загрузка отношений (Relations) запрета поворотов критически важна для построения корректного графа (учет знаков <<Движение только прямо>>).
    \item \textbf{Barriers:} Точечные объекты (шлагбаумы, ворота), которые могут физически блокировать проезд, загружаются как отдельные узлы для последующего учета в весовой функции.
\end{enumerate}

\section{Алгоритм загрузки данных}

Система реализует <<ленивую>> (lazy) стратегию загрузки: данные скачиваются только для тех областей, которые запрашиваются пользователем. Для координации процесса используется механизм тайловой нарезки.

\subsection{Плиточная декомпозиция}
Глобальная область карты разбивается на тайлы размером $0.05^\circ \times 0.05^\circ$. При запросе данных происходит проверка наличия соответствующего тайла в базе `osm.cached\_tiles` (рисунок~\ref{fig:tile-loading}).

\begin{figure}[H]
    \centering
    \includemermaid[width=0.55\linewidth]{02-tile-loading}
    \caption{Алгоритм загрузки и обработки тайлов}
    \label{fig:tile-loading}
\end{figure}

Особенности реализации:
\begin{itemize}
    \item \textbf{Идемпотентность:} Использование конструкции `INSERT ... ON CONFLICT DO UPDATE` гарантирует, что повторная загрузка одной области (например, при обновлении карты) корректно обновит существующие записи, не создавая дубликатов.
    \item \textbf{Rate Limiting:} Клиентские запросы к Overpass API буферизируются через `asyncio.Semaphore`, ограничивая количество одновременных соединений (не более 5) для соблюдения политики использования сервиса (Fair Use Policy).
\end{itemize}

\subsection{Контроль качества данных}
При импорте выполняется валидация геометрии:
\begin{itemize}
    \item Исключаются объекты с пустой или невалидной геометрией (менее 2 точек).
    \item Нормализуются теги ограничений скорости (`maxspeed`): значения `ru:urban` или `none` преобразуются в числовые эквиваленты (60 км/ч).
    \item Обрабатываются ошибки связности: дороги, не соединенные с основным графом, помечаются при анализе топологии.
\end{itemize}
