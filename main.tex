% main.tex
% \documentclass[draft]{bmstu-iu1}  % 1. Загружаем класс (ищется в lib/bmstu)
\documentclass{bmstu-iu1}  % 1. Загружаем класс (ищется в lib/bmstu)
\usepackage{project-setup} % 2. Загружаем наши настройки (ищется в lib)

% Подключаем библиографию
\addbibresource{chapters/biblio.bib} 

\begin{document}

% Титульный лист для НИР (6 аргументов)
% Порядок аргументов обычно такой:
% 1. Факультет
% 2. Кафедра
% 3. Тема работы
% 4. Группа
% 5. Студент
% 6. Руководитель

\makeresearchtitle
    {Информатика, искусственный интеллект и системы управления} % Название факультета
    {Искусственный интеллект в киберфизических системах} % Название кафедры
    {Программная система формирования маршрутов городского транспорта} % Тема работы
    {Иванов~В.~И./ИУ12-11М} % Номер группы/ФИО студента (если авторов несколько, их необходимо разделить запятой)
    {Масленников~А.~Л.} % ФИО научного руководителя
    {} % ФИО консультанта (необязательный аргумент; если консультантов несколько, их необходимо разделить запятой)

% Вставляем ТЗ. Оно займет свои страницы. 
% Нумерация в дипломе "проглотит" эти страницы (учтет их количество), 
% но цифры на них печатать не будет.
\insertTask{assets/pdf/ИУ12-11М_Иванов_В.И._ТЗ_НИР-1.pdf}

\maketableofcontents

% Список сокращений
% chapters/00-abbreviations.tex

\begin{abbreviations}
    \definition{БД}{база данных}
    \definition{СУБД}{система управления базами данных}
    \definition{API}{Application Programming Interface (интерфейс программирования приложений)}
    \definition{JSONB}{JSON Binary}
    \definition{LOD}{Level of Detail (уровень детализации)}
    \definition{MVT}{Mapbox Vector Tile (векторный тайл)}
\end{abbreviations}

% Основные главы
\chapter*{ВВЕДЕНИЕ}
\addcontentsline{toc}{chapter}{ВВЕДЕНИЕ}

\textbf{Актуальность темы} заключается в том, что развитие автономного транспорта и мультиагентных логистических систем предъявляет качественно новые требования к алгоритмам построения маршрутов. Традиционные навигационные решения, такие как OSRM или GraphHopper, ориентированы на достижение минимальной задержки запроса за счет использования статических структур данных (иерархии сокращений -- Contraction Hierarchies). Однако в условиях мегаполиса, характеризующегося высокой динамикой дорожной ситуации (аварии, ремонтные работы, оперативное перекрытие зон), такие системы становятся неэффективными: время перестройки статических индексов для графа масштаба московской агломерации составляет от 15 до 60 минут. Любое локальное изменение графа требует полной ревалидации данных, что создает критическую задержку в управлении транспортными потоками. В этой связи актуальной задачей является разработка навигационных комплексов на базе архитектуры с центрированием на базе данных, обеспечивающих мгновенное обновление весов ребер графа (мгновенная актуализация дорожной ситуации) при сохранении высокой точности и вариативности маршрутов.

\textbf{Объектом исследования} являются алгоритмы и программные средства обеспечения навигации в динамически изменяющихся графовых структурах.

\textbf{Предметом исследования} является архитектура и программная реализация распределенной навигационной системы для мультиагентных транспортных комплексов.

\textbf{Цель работы} -- разработка и исследование модульного программного комплекса для поиска и оптимизации маршрутов в динамической среде с использованием реляционной модели хранения графа.

Для достижения поставленной цели необходимо решить следующие \textbf{задачи}:
\begin{enumerate}
    \item проанализировать методы маршрутизации в динамических графах и обосновать выбор СУБД-ориентированной архитектуры для систем реального времени;
    \item разработать математическое и алгоритмическое обеспечение, включающее методы пространственного секционирования, динамического ограничения области поиска и итеративных штрафов, для эффективной навигации в сверхбольших дорожных сетях;
    \item спроектировать и реализовать микросервисную архитектуру программного комплекса, обеспечивающую асинхронное взаимодействие компонентов и горизонтальную масштабируемость;
    \item разработать кроссплатформенное клиентское приложение с гибридным механизмом рендеринга векторных карт для визуализации маршрутов и управления параметрами системы;
    \item провести экспериментальное исследование производительности системы и качества маршрутизации на реальных данных Московской агломерации.
\end{enumerate}

\textbf{Методы исследования.} При решении поставленных задач использовались методы теории графов (алгоритмы поиска кратчайших путей), методы системного анализа и проектирования программного обеспечения, принципы организации реляционных баз данных и геоинформационных систем (ГИС), а также методы статистического анализа данных.

\textbf{Научная новизна} работы заключается в разработке комплексного алгоритмического обеспечения на базе СУБД PostgreSQL. В отличие от существующих статических систем, предложенная интеграция метода пространственного секционирования, механизм динамического ограничения области поиска и алгоритма итеративных штрафов позволила не только снизить потребление оперативной памяти при построении графа с 12,4~ГБ до 370~МБ, но и обеспечила принципиальную возможность мгновенной актуализации весов ребер в реальном времени без пересчета вспомогательных структур. % "вспомогательных структур" -- Я имею в виду индексы алгоритма Contraction Hierarchies и графовые тайлы. OSRM строит их час, а моя система обходится без них, поэтому обновляет пробки мгновенно

\textbf{Практическая значимость} работы состоит в создании масштабируемой программной платформы для формирования маршрутов городского транспорта. Разработанная инфраструктура является технологическим фундаментом для развертывания мультиагентных систем управления: благодаря поддержке динамического обновления весов через транзакционные механизмы СУБД, система готова к интеграции модулей мониторинга пробок и оперативного диспетчерского управления без изменения архитектурного ядра.

\chapter{Исследовательский раздел}
\label{ch:analysis}

\section{Анализ предметной области и постановка задачи}

Современные навигационные системы мультиагентного типа предъявляют высокие требования к гибкости маршрутизации, качеству визуализации и масштабируемости архитектуры. В условиях растущей нагрузки и необходимости обработки данных в реальном времени, традиционные монолитные архитектуры демонстрируют ряд существенных ограничений. В частности, они обладают низкой отказоустойчивостью и сложностью в обновлении отдельных компонентов без остановки всей системы. В рамках данной работы рассматривается проектирование распределенной системы, в которой ключевым архитектурным паттерном является API Gateway, выполняющий функцию единой точки входа (Single Entry Point).

Архитектура системы построена по принципу микросервисов, где каждый компонент отвечает за изолированную область бизнес-логики. Такое разделение позволяет использовать различные технологические стеки, оптимально подходящие для конкретных задач, а также независимо масштабировать сервисы в зависимости от нагрузки.

Система состоит из следующих основных компонентов:
\begin{itemize}
    \item \textbf{Сервис маршрутизации (Router Service)} --- модуль, обеспечивающий вычисление оптимальных маршрутов и управление дорожным графом. Он отвечает за алгоритмическую часть поиска пути.
    \item \textbf{Сервис обработки данных (Data Processor Service)} --- компонент, осуществляющий взаимодействие с внешними источниками геоданных, в частности OpenStreetMap (OSM), и генерацию векторных тайлов (MVT) в режиме реального времени.
    \item \textbf{Клиентское приложение (Qt Client)} --- интерфейс пользователя, реализованный на базе фреймворка Qt, обеспечивающий визуализацию картографической информации и взаимодействие с системой.
\end{itemize}

Общая схема архитектуры системы представлена на рисунке~\ref{fig:system-arch}.

\begin{figure}[H]
    \centering
    \includemermaid[width=0.55\linewidth]{01-system-arch}
    \caption{Схема архитектуры системы (C4 Context)}
    \label{fig:system-arch}
\end{figure}

Использование паттерна Gateway позволяет скрыть внутреннюю топологию сети от клиента, предоставляя единый унифицированный интерфейс API. Взаимодействие осуществляется исключительно через шлюз, который маршрутизирует запросы к соответствующим внутренним сервисам (upstream). Данный подход обеспечивает изоляцию внутренних компонентов --- клиент не имеет прямого доступа к сервисам маршрутизации или обработки данных, что повышает безопасность системы. Кроме того, это упрощает клиентскую логику, так как клиенту не нужно знать адреса и протоколы взаимодействия с каждым отдельным сервисом, и обеспечивает возможность горизонтального масштабирования инфраструктуры.

В классических реализациях микросервисных шлюзов часто встречается проблема жестко заданных маршрутов (Hardcoded Routes). При таком подходе правила маршрутизации фиксируются непосредственно в исходном коде программного обеспечения. Конфигурация оказывается распределенной по различным модулям, а добавление нового микросервиса требует изменения логики шлюза, перекомпиляции и повторного развертывания. Это существенно замедляет процесс разработки и внедрения новых функций.

Для обеспечения эксплуатационной гибкости и устранения недостатков жесткой конфигурации, к инфраструктуре предъявляются следующие требования:
\begin{enumerate}
    \item \textbf{Принцип DRY (Don't Repeat Yourself).} Логика проксирования должна быть универсальной и не зависеть от специфики конкретного сервиса.
    \item \textbf{Декларативность.} Маршруты должны описываться в конфигурационных файлах (в формате YAML), а не в программном коде. Это позволяет изменять топологию системы без пересборки приложения.
    \item \textbf{Производительность.} Накладные расходы (overhead) на проксирование запросов не должны оказывать существенного влияния на общее время отклика системы и не должны превышать 10 мс.
\end{enumerate}

\section{Сравнительный анализ технологий маршрутизации}

Ключевой функциональной задачей разрабатываемой системы является поиск оптимального пути по дорожному графу. Для масштабов крупной агломерации, такой как Москва, граф содержит более 250 000 ребер. Специфическим требованием, отличающим данную систему от классических навигаторов, выступает возможность динамического изменения весов ребер. Это необходимо для учета перекрытий дорог, изменения дорожной обстановки или введения ограничений для спецтранспорта в режиме реального времени без необходимости полной перестройки графа.

Существующие индустриальные стандарты, такие как OSRM (Open Source Routing Machine), Valhalla и GraphHopper, достигают субсекундной задержки (менее 10 мс) за счет использования алгоритмов на базе иерархий сжатия (Contraction Hierarchies --- CH).

Стоит отметить, что первоначальные попытки использования классического алгоритма A* (A-Star) на базе реляционной СУБД показали неудовлетворительные результаты. Основной причиной стала не вычислительная сложность самого алгоритма, а накладные расходы на извлечение данных. Хранение графа в нормализованной форме (таблицы `nodes` и `edges`) требует выполнения операции `JOIN` для получения координат каждой вершины в процессе обхода. При длине маршрута в 1000 ребер это приводит к тысячам дисковых операций чтения (Random Seek), что увеличивает время поиска до 30--40 секунд даже при наличии индексов.

Метод CH, напротив, заключается в предварительной обработке графа и создании дополнительных <<ярлыков>> (shortcuts) между узлами, которые позволяют пропускать множество промежуточных вершин при поиске пути. Данный подход обеспечивает высокую скорость поиска в статичном графе.

Однако критическим ограничением метода Contraction Hierarchies является его статичность. Изменение веса даже одного ребра нарушает целостность иерархии ярлыков и требует полной перестройки индексов. Для графа масштаба Москвы данный процесс занимает от 30 до 60 минут процессорного времени. Это делает невозможным учет динамических факторов в режиме реального времени, так как за время перестройки графа дорожная ситуация может измениться вновь.

Сравнение существующих технологий маршрутизации по ключевым критериям приведено в таблице~\ref{tab:routing-comparison}.

\begin{table}[H]
    \centering
    \caption{Сравнение технологий маршрутизации}
    \label{tab:routing-comparison}
    \begin{tabular}{|p{0.22\linewidth}|p{0.18\linewidth}|p{0.18\linewidth}|p{0.18\linewidth}|p{0.14\linewidth}|}
        \hline
        \textbf{Критерий} & \textbf{OSM} & \textbf{Valhalla} & \textbf{GraphHopper} & \textbf{pgRouting} \\ \hline
        Архитектура хранения & In-Memory (RAM) & MMap Tiles & In-Memory (JVM) & PostgreSQL (Disk) \\ \hline
        Предобработка & Contraction Hierarchies & Обязательна & CH / Landmarks & Нет \\ \hline
        Время перестройки & 30--60 мин & 20--40 мин & 15--30 мин & 0 мин \\ \hline
        Динамика весов & Невозможно & Ограничено & Невозможно & Мгновенно \\ \hline
        Средняя задержка & $< 10$ мс & 15--30 мс & 20--50 мс & 2500--6000 мс \\ \hline
        Потребление RAM & 4--8 ГБ & 2--4 ГБ & 6--12 ГБ & 370 МБ \\ \hline
    \end{tabular}
\end{table}

На основе проведенного анализа было принято решение использовать технологический стек \textbf{PostgreSQL + PostGIS + pgRouting}. Несмотря на более низкую производительность «чистого» поиска пути по сравнению с решениями In-Memory, данный выбор обусловлен следующими факторами, критичными для динамической системы:
\begin{itemize}
    \item \textbf{Отсутствие времени перестройки.} Веса ребер хранятся непосредственно в таблицах базы данных. Изменение веса ребра представляет собой стандартную SQL-транзакцию \texttt{UPDATE}, которая выполняется за миллисекунды и мгновенно учитывается при следующем поиске маршрута.
    \item \textbf{Гибкость логики.} Ограничения движения любой сложности (временные окна, типы транспортных средств, габариты) реализуются посредством изменения SQL-запросов, без необходимости модификации движка маршрутизации.
    \item \textbf{Эффективность использования памяти.} Граф хранится на диске, а в оперативной памяти (в области Shared Buffers) находятся только часто используемые участки данных ("горячие" данные). Это позволяет работать с огромными графами на оборудовании с ограниченным объемом RAM.
\end{itemize}

Компромиссом выбранного решения является увеличение времени поиска маршрута до единиц секунд, что допустимо для логистических задач планирования. Для обеспечения отзывчивости интерфейса данная задержка компенсируется реализацией асинхронного взаимодействия на стороне клиента.

\section{Архитектура подсистемы обработки геоданных}

Модуль обработки геоданных (Data Processor) решает комплекс задач, связанных с получением актуальной картографической информации. Основными функциями являются загрузка сырых данных из OpenStreetMap и генерация векторных тайлов (MVT --- Mapbox Vector Tiles) для отображения карты на клиенте. Обеспечение актуальности данных критически важно для корректной маршрутизации.

Визуализация векторных данных требует эффективной стратегии нарезки геометрии на тайлы. Существуют различные подходы к решению этой задачи, от полной предварительной генерации до формирования тайлов в момент запроса. Были рассмотрены основные стратегии, представленные в таблице~\ref{tab:tiles-comparison}.

\begin{table}[H]
    \centering
    \caption{Сравнение подходов к генерации векторных тайлов}
    \label{tab:tiles-comparison}
    \begin{tabular}{|p{0.3\linewidth}|p{0.3\linewidth}|p{0.3\linewidth}|}
        \hline
        \textbf{Подход} & \textbf{Преимущества} & \textbf{Недостатки} \\ \hline
        Статическая генерация (Tippecanoe) & Высокая скорость отдачи (раздача статичных файлов через Nginx/CDN) & Невозможность динамической смены стилей, длительный процесс полной пересборки при обновлении данных \\ \hline
        Сервер тайлов (Tileserver GL) & Готовое коробочное решение, простота развертывания & Жесткая привязка к формату MBTiles, сложность кастомизации логики \\ \hline
        Генерация по запросу (On-Demand: PostGIS + ST\_AsMVT) & Полный контроль логики через SQL, динамическая детализация (LOD), отсутствие дублирования данных & Повышенная нагрузка на БД, более низкая скорость ответа по сравнению со статикой \\ \hline
    \end{tabular}
\end{table}

В результате анализа был выбран подход \textbf{On-Demand Generation} (генерация по запросу). В данной архитектуре тайлы не хранятся на диске в виде файлов, а формируются динамически SQL-запросом с использованием функции \texttt{ST\_AsMVT} в момент обращения клиента. Это позволяет реализовать концепцию динамического уровня детализации (Level of Detail --- LOD): состав объектов в тайле (например, отображение дворовых проездов только на крупных масштабах или скрытие мелких деталей на обзорных картах) изменяется путем модификации параметров запроса. Такой подход исключает необходимость длительной перегенерации кэша при изменении стилей отображения или обновлении атрибутов дорог.

\section{Архитектура клиентского приложения}

Клиентское приложение является основным интерфейсом взаимодействия пользователя с системой. Оно должно обеспечивать не только ввод пунктов назначения и просмотр маршрутов, но и качественную, плавную визуализацию карты с высокой частотой кадров (60 FPS), поддерживая при этом интерактивность.

Для реализации клиента был выбран гибридный архитектурный подход, объединяющий возможности нативного фреймворка \textbf{PyQt5} и современного веб-движка для карт \textbf{MapLibre GL JS}. Сравнение рассмотренных вариантов реализации приведено в таблице~\ref{tab:client-arch}.

\begin{table}[H]
    \centering
    \caption{Сравнение подходов к реализации клиента}
    \label{tab:client-arch}
    \begin{tabular}{|p{0.3\linewidth}|p{0.3\linewidth}|p{0.3\linewidth}|}
        \hline
        \textbf{Подход} & \textbf{Преимущества} & \textbf{Недостатки} \\ \hline
        Нативный Qt (QGraphicsView) & Высокая производительность C++, полный контроль над управлением памятью & Отсутствие встроенной поддержки формата MVT, высокая трудоемкость реализации стилизации карт \\ \hline
        Web-приложение (React/Vue) & Простая разработка UI, богатая экосистема картографических библиотек & Ограничения браузерной песочницы (доступ к локальным файлам, оборудованию), зависимость от браузера \\ \hline
        Гибридное приложение (Qt + WebView) & Баланс производительности GPU (WebGL через WebView) и возможностей Desktop-приложения & Накладные расходы памяти на запуск браузерного движка, сложность реализации моста JS-Python \\ \hline
    \end{tabular}
\end{table}

Архитектура клиентского приложения построена на строгом разделении ответственности между компонентами:
\begin{enumerate}
    \item \textbf{Слой Python (PyQt5)} управляет жизненным циклом приложения, потоками (QThread), сетевым взаимодействием и основной бизнес-логикой. Он отвечает за взаимодействие с сервером и обработку данных.
    \item \textbf{Слой JavaScript (MapLibre GL JS)}, работающий внутри компонента \texttt{QWebEngineView}, отвечает исключительно за рендеринг векторных тайлов с использованием WebGL. Это позволяет задействовать аппаратное ускорение графического процессора для плавной отрисовки карты.
    \item \textbf{Мост QWebChannel} обеспечивает двунаправленную асинхронную связь между средой Python и JavaScript, позволяя передавать команды управления картой и получать события от пользователя.
\end{enumerate}

Данный подход позволяет использовать современные стандарты веб-картографии (MVT, шейдеры WebGL) в рамках функционального десктопного приложения.

\chapter{Математическое и алгоритмическое обеспечение системы}

\section{Математическая модель дорожной сети}

В основе навигационной системы лежит представление дорожной сети в виде взвешенного ориентированного графа $G = (V, E)$, где $V$ -- множество вершин, соответствующих перекресткам и тупикам, а $E$ -- множество ребер, представляющих участки дорог. Каждое ребро $e_{ij} \in E$, соединяющее вершины $v_i$ и $v_j$, характеризуется весовой функцией $w(e_{ij}, \tau)$, зависящей от времени $\tau$.

Целевая функция поиска оптимального маршрута $P^*$ формулируется как минимизация суммарной стоимости прохождения ребер маршрута\cite{dijkstra1959_yale}:
\begin{equation}
P^* = \arg \min_{P \in \mathcal{P}_{st}} \sum_{e \in P} w(e, \tau),
\end{equation}
где $\mathcal{P}_{st}$ -- множество всех допустимых путей из точки $s$ в точку $t$.

Весовая функция ребра $w(e, t)$ рассчитывается как прогнозируемое время прохождения участка с учетом динамических факторов:
\begin{equation}
    w(e, t) = \frac{L_e}{v_{eff}(e, t)},
\end{equation}
\where{
    $L_e$           & физическая длина участка дороги (в метрах); \\
    $v_{eff}(e, t)$ & эффективная скорость на участке в момент времени $t$.
}

Эффективная скорость моделируется на основе фундаментальной диаграммы транспортного потока (Гриншилдса). В упрощенном виде она зависит от коэффициента загруженности $K_{load}$:
\begin{equation}
    v_{eff} = v_{max} \cdot \eta_{surf} \cdot K_{load}(\rho),
\end{equation}
\where{
    $v_{max}$    & максимально допустимая скорость на участке (км/ч); \\
    $\eta_{surf}$ & коэффициент типа дорожного покрытия ($1.0$ -- асфальт, $0.6$ -- грунт); \\
    $K_{load}(\rho)$ & функция зависимости скорости от плотности потока.
}

Плотность потока $\rho$ нормируется относительно пропускной способности ребра, зависящей от числа полос $N_{lanes}$:
\begin{equation}
    \rho = \frac{N_{agents} \cdot L_{agent}}{L_e \cdot N_{lanes}},
\end{equation}
\where{
    $N_{agents}$ & текущее количество агентов на ребре; \\
    $L_{agent}$  & средняя длина транспортного средства (5 м); \\
    $N_{lanes}$  & количество полос движения.
}

При малой плотности ($\rho \to 0$) агенты движутся с максимальной разрешенной скоростью. При приближении к критической плотности (затор) скорость снижается до минимального порогового значения (например, 5-10 км/ч), что отражает нелинейный характер образования пробок\cite{greenshields1935_traffic}.
Данная модель позволяет системе динамически перераспределять маршруты, избегая участков с исчерпанной пропускной способностью.

\section{Плиточная декомпозиция графа для оптимизации ресурсов}

\subsection{Проблема исчерпания ресурсов при построении топологии}
Одной из ключевых технических проблем при обработке графа Московской агломерации (более 250 000 ребер) стало исчерпание оперативной памяти (Out Of Memory) при выполнении стандартной процедуры построения топологии \texttt{pgr\_nodeNetwork}. Данная функция пытается загрузить всю геометрию в память для поиска пересечений, что приводило к потреблению более 12 ГБ ОЗУ и аварийному завершению процесса (Exit Code 137). 

Анализ исходных данных выявил наличие сверхдлинных ребер (например, участки МКАД длиной до 8373 м), которые создавали «комбинаторный взрыв» при вычислении пересечений с локальной дорожной сетью.

\subsection{Алгоритм пространственного секционирования}
Для решения проблемы был разработан алгоритм пространственного секционирования графа (Grid Partitioning). Метод заключается в декомпозиции исходной области карты на независимые квадратные наборы (тайлы) фиксированного размера.

Процесс обработки включает следующие этапы:
\begin{enumerate}
    \item \textbf{Секционирование.} Вся область карты разбивается на тайлы размером $0.05^\circ \times 0.05^\circ$ (приблизительно $3 \times 5$ км).
    \item \textbf{Предварительная нарезка.} Применяется функция \texttt{ST\_Subdivide} для принудительного разбиения длинных геометрий на сегменты не более 500 метров\cite{postgis_vision_2018}. Это гарантирует, что ребро не будет пересекать более 4 смежных тайлов.
    \item \textbf{Локальная топологизация.} Построение узловой сети выполняется независимо для каждого тайла, что позволяет обрабатывать граф по частям, не выходя за пределы доступной оперативной памяти.
\end{enumerate}

В результате внедрения алгоритма максимальная длина ребра в графе сократилась с 8373 м до 871 м, что существенно повысило точность аппроксимации маршрутов. Сравнительный анализ эффективности подходов приведен в таблице~\ref{tab:oom-comparison}.

\begin{table}[h]
    \caption{Сравнение ресурсов при построении графа}
    \label{tab:oom-comparison}
    \begin{tabular}{|p{0.35\textwidth}|p{0.25\textwidth}|p{0.3\textwidth}|}
    \hline
    \textbf{Метрика} & \textbf{Стандартный подход} & \textbf{Секционирование (Наш метод)} \\ \hline
    Потребление памяти (RAM) & $> 12.4$ ГБ (сбой) & \textbf{370~МБ} \\ \hline
    Время выполнения & $> 2$ часов & \textbf{13 минут} \\ \hline
    Максимальная длина ребра & 8373 м & 871 м \\ \hline
    Результат & Аварийное завершение & Успешное построение \\ \hline
    \end{tabular}
\end{table}

\section{Алгоритм формирования альтернативных маршрутов}

\subsection{Критика эвристических алгоритмов в реляционных СУБД}
Классическим подходом к ускорению поиска пути является алгоритм A* (A-Star), использующий эвристическую функцию $h(v)$ для оценки расстояния до цели\cite{hart1968_scirp, astar_bstu_neuro, hart1968_auckland, hart1968_stfx}. Однако эксперименты показали его неэффективность в архитектуре Database-Centric\cite{hart1968_scienceopen}.

В реляционной модели координаты вершин хранятся в отдельной нормализованной таблице. Вычисление евклидовой эвристики $h(v) = \sqrt{(x_t - x_v)^2 + (y_t - y_v)^2}$ требует выполнения операции соединения таблиц базы данных (JOIN) на каждой итерации алгоритма. Это приводило к деградации производительности: время поиска маршрута длиной 30 км достигало 47~секунд, где 90~\% времени занимали накладные расходы на извлечение координат\cite{hart1968_semantic}.

\subsection{Метод итеративных штрафов}
Вместо A* и классического алгоритма Йена (K-Shortest Paths), который часто возвращает топологически идентичные пути\cite{yen1971_mit, yen1971_ideas, eppstein1998_siam, yen1971_adrian, wiki_ksp_ru}, был применен метод итеративных штрафов (Iterative Penalty)\cite{roditty2012_replacement}, позволяющий также учитывать динамику среды\cite{iet_ppo_dynamic_2021}.

Алгоритм работает следующим образом:
\begin{enumerate}
    \item Находится кратчайший путь с текущими весами.
    \item Веса ребер $e$, входящих в найденный путь, увеличиваются согласно рекуррентной формуле:
    \begin{equation}
    w_{k}(e) = w_{k-1}(e) \cdot \mu,
    \end{equation}
    \where{
        $\mu = 5.0$ & эмпирический коэффициент штрафа.
    }
    \item Процедура повторяется $K$ раз.
\end{enumerate}

Такой подход вынуждает алгоритм искать маршруты, проходящие по принципиально другим улицам (например, набережная вместо проспекта), обеспечивая семантическое разнообразие альтернатив.

\subsection{Динамическое ограничение области поиска}
Для оптимизации производительности алгоритма Дейкстры используется метод динамического ограничивающего прямоугольника (Dynamic Bounding Box). Поиск выполняется не по всему графу, а только среди ребер, попадающих в область, определяемую формулой\cite{dijkstra1959_eudml}:
\begin{equation}
\delta(d) = \max(0.015^\circ, d \cdot 0.3),
\end{equation}
\where{
    $d$ & евклидово расстояние между точками старта и финиша.
} 

Данная эвристика позволяет отсечь до 90~\% нерелевантных ребер графа, сокращая время поиска с десятков секунд до 0.9–2.1 секунды, сохраняя при этом гарантию нахождения оптимального пути (при достаточном коэффициенте запаса 0.3).

\begin{figure}[H]
    \centering
    \includegraphics[width=0.95\linewidth]{components_map}
    \caption{Карта связности графа (Strongly Connected Components). Различными цветами выделены изолированные подграфы, не имеющие выхода на основную дорожную сеть.}
    \label{fig:components-map}
\end{figure}

\section{Выводы по второй главе}

Разработанная математическая модель и алгоритмическое обеспечение позволяют эффективно решать задачу маршрутизации в условиях ограничений реляционной СУБД. Применение алгоритма пространственного секционирования устранило проблему исчерпания оперативной памяти, снизив потребление с 12~ГБ до 370~МБ. Отказ от алгоритма A* в пользу оптимизированного алгоритма Дейкстры с динамическим ограничением области поиска позволил достичь приемлемой производительности без усложнения архитектуры хранения данных.

\chapter{Реализация подсистемы маршрутизации}
\label{ch:implementation}

В данном разделе рассматриваются ключевые алгоритмические и архитектурные решения, позволившие реализовать масштабируемую систему маршрутизации на базе PostgreSQL. Основное внимание уделено двум аспектам: эффективному построению дорожного графа в условиях ограниченных ресурсов и реализации гибридного алгоритма поиска пути.

\section{Архитектура хранения пространственных данных}

Для обеспечения целостности данных и оптимизации производительности была разработана двухуровневая схема хранения. Данные OpenStreetMap (OSM) хранятся в <<сыром>> виде для обеспечения возможности обновлений, в то время как маршрутный граф представляет собой оптимизированную топологическую структуру.

Схема взаимодействия таблиц представлена на диаграмме (рисунок~\ref{fig:graph-schema}).

\begin{figure}[H]
    \centering
    \includemermaid{03-graph-schema}
    \caption{ER-диаграмма схем OSM (источник) и GRAPHS (топология)}
    \label{fig:graph-schema}
\end{figure}

Ключевой особенностью является сохранение связи (Data Lineage) между ребром графа (`graphs.edges`) и исходной линией (`osm.ways`) через внешний ключ. Это позволяет динамически подтягивать любые атрибуты (например, время работы или ограничения по весу) без необходимости перестройки топологии.

\section{Процесс построения дорожного графа}

Построение графа из сырых геоданных — вычислительно сложная задача, требующая поиска всех пересечений дорог (Noding). Классический подход загрузки всех данных в оперативную память оказался неприменим для дорожной сети Москвы (более 250 000 рёбер), вызывая переполнение памяти (OOM) на серверах с 16 ГБ RAM.

\subsection{Алгоритм Grid Partitioning}

Для решения проблемы OOM был разработан алгоритм плиточной нарезки (Grid Partitioning). Идея заключается в декомпозиции глобальной задачи на множество локальных подзадач, решаемых независимо.

Общая схема процесса построения представлена на рисунке~\ref{fig:build-flow}.

\begin{figure}[H]
    \centering
    \includemermaid[width=0.55\linewidth]{03-build-flow}
    \caption{Алгоритм построения графа}
    \label{fig:build-flow}
\end{figure}

Процесс состоит из следующих этапов.

\textbf{Этап 1. Препроцессинг и фильтрация слоёв.}
Перед нарезкой выполняется разделение геометрий на группы. Дороги, проходящие на разных уровнях (мосты, тоннели), не должны образовывать перекрестков с дорогами на земле.
\begin{enumerate}
    \item \textbf{Ground Layer:} Обычные дороги (`layer=0`). Подлежат полной топологической обработке.
    \item \textbf{Isolated Layer:} Мосты и тоннели (`bridge=yes`, `tunnel=yes`). Исключаются из поиска пересечений, чтобы избежать создания ложных узлов в местах визуального наложения линий в 2D-проекции.
\end{enumerate}
Кроме того, применяется функция `ST\_Subdivide` для разбиения длинных геометрий (например, МКАД длиной 109 км) на сегменты не более 500 метров. Это улучшает эффективность пространственных индексов.

\textbf{Этап 2. Параллельная обработка тайлов.}
Территория разбивается на тайлы размером $0.05^\circ \times 0.05^\circ$. Обработка каждого тайла происходит независимо (рисунок~\ref{fig:grid-partition}), что позволяет линейно масштабировать процесс по количеству ядер CPU.

\begin{figure}[H]
    \centering
    \includemermaid{03-grid-partition}
    \caption{Параллельная обработка тайлов}
    \label{fig:grid-partition}
\end{figure}

Внутри каждого тайла выполняется функция `pgr\_nodeNetwork` в оперативной памяти процесса Python. Это снизило потребление RAM с 12 ГБ до 370 МБ и сократило время полной сборки графа с 4 часов до 13 минут.

\textbf{Этап 3. Наследование атрибутов.}
В процессе нарезки (`ST\_Node`) исходные линии дробятся на мелкие сегменты, теряя свои атрибуты. Для восстановления данных реализован алгоритм пространственного сопоставления:
\begin{equation}
L_{overlap} = \text{Length}(\text{Intersection}(E_{new}, E_{old}))
\label{eq:overlap}
\end{equation}
Если $L_{overlap} > 0.8 \cdot \text{Length}(E_{new})$, то сегмент $E_{new}$ наследует название улицы, скоростной режим и тип дороги от $E_{old}$. Это фильтрует случайные пересечения.

\subsection{Анализ связности графа}
После построения выполняется проверка на наличие изолированных <<островов>> — участков дорожной сети, не имеющих выхода в основную компоненту (закрытые поселки, ошибки картографии).
Анализ с помощью `pgr\_connectedComponents` показал, что основная компонента связности охватывает 98.4\% всех ребер, что соответствует критериям качества для навигационных систем. Оставшиеся 1.6\% представляют собой действительно изолированные зоны или ошибки данных OSM, которые исключаются из маршрутизации.

\section{Реализация алгоритмов поиска пути}

Ядро маршрутизации реализовано на уровне SQL-функций PostgreSQL, что минимизирует накладные расходы на передачу данных между диском и приложением.

\subsection{Жизненный цикл запроса}
Обработка пользовательского запроса проходит через конвейер фильтрации и постобработки (рисунок~\ref{fig:runtime-flow}).

\begin{figure}[H]
    \centering
    \includemermaid[width=0.55\linewidth]{03-runtime-flow}
    \caption{Конвейер обработки запроса маршрутизации}
    \label{fig:runtime-flow}
\end{figure}

\textbf{Привязка к графу (Smart Snapping).}
Вместо поиска ближайшего ребра, который может привести к ошибкам на многоуровневых развязках (привязка к туннелю под землей), используется поиск $K$ ближайших узлов (Vertex Snapping) с помощью KNN-индекса (`<->` operator в PostGIS). Это гарантирует, что маршрут начнется с валидной точки (перекрестка или примыкания).

\textbf{Динамическое ограничение области (Dynamic BBOX).}
Для оптимизации производительности поиск выполняется не по всему графу (250 тыс. ребер), а только внутри ограничивающего прямоугольника, охватывающего старт и финиш, с буфером 30\%:
\begin{equation}
Buffer = \max(1.5 \text{ km}, 0.3 \cdot \text{Distance}(Start, End))
\label{eq:bbox-buffer}
\end{equation}
Это сократил время поиска для локальных маршрутов с 3.2 с до 0.9 с.

\subsection{Алгоритм Iterative Penalty Dijkstra}
Для построения альтернативных маршрутов ($K$-Shortest Paths) используется метод итеративных штрафов, так как стандартный алгоритм Йена (Yen's Algorithm) склонен генерировать пути с незначительными отличиями (микро-объезды).

Алгоритм работает следующим образом:
\begin{enumerate}
    \item Найти кратчайший путь $R_1$ с помощью алгоритма Дейкстры.
    \item Увеличить стоимость всех ребер, входящих в $R_1$, в 5 раз: $Cost(e) \leftarrow Cost(e) \cdot 5.0$.
    \item Запустить поиск снова для нахождения $R_2$.
    \item Повторить $K$ раз.
\end{enumerate}
Коэффициент штрафа 5.0 подобран эмпирически: он заставляет алгоритм искать принципиально иные пути (например, через другую магистраль), но не приводит к абсурдным объездам.

\subsection{Коррекция геометрии (Zig-Zag Fix)}
В исходных данных OSM линии имеют направление оцифровки. Если маршрут проходит <<против шерсти>> двусторонней дороги, сырая геометрия будет инвертирована. Для устранения артефактов визуализации (<<зубья пилы>>) применяется динамический разворот геометрии:
\begin{code}{zigzag-fix}{SQL}{Коррекция геометрии (Zig-Zag Fix)}
CASE 
    WHEN op.node = edge.source THEN edge.geometry 
    ELSE ST_Reverse(edge.geometry) 
END
\end{code}
Это обеспечивает непрерывность линии маршрута на клиенте.

\section{Оптимизация потребления ресурсов}

Одной из критических проблем, выявленных на этапе прототипирования, стало исчерпание оперативной памяти (OOM) при попытке построить граф дорожной сети Москвы целиком. Классический подход требовал загрузки всех 250 000 ребер в память для построения топологии.

Применение разработанного алгоритма Grid Partitioning позволило радикально снизить требования к оборудованию. Сравнительные характеристики приведены в таблице~\ref{tab:oom-comparison}.

\begin{table}[H]
    \centering
    \caption{Сравнение потребления ресурсов (Legacy vs Grid Partitioning)}
    \label{tab:oom-comparison}
    \begin{tabular}{|l|c|c|c|}
        \hline
        \textbf{Параметр} & \textbf{Монолитный подход} & \textbf{Grid Partitioning} & \textbf{Эффект} \\ \hline
        Пиковое RAM & 12.4 ГБ & 370 МБ & $\downarrow$ 33.5x \\ \hline
        Время сборки & 4 ч 15 мин & 13 мин & $\downarrow$ 19.6x \\ \hline
        Стабильность & OOM Crash & 100\% Success & Решено \\ \hline
    \end{tabular}
\end{table}

\section{Реализация подсистемы Gateway}

Шлюз (Gateway) играет роль единой точки входа и обеспечивает прозрачное проксирование WebSocket-соединений. Реализация основана на асинхронном фреймворке \texttt{FastAPI} и библиотеке \texttt{websockets}.

\subsection{Проксирование WebSocket}
Для обеспечения дуплексной связи между клиентом и сервисом маршрутизации используется механизм <<трубы>> (pipe), реализованный через \texttt{asyncio.gather}. Шлюз одновременно запускает две задачи: чтение от клиента с отправкой в сервис и чтение от сервиса с отправкой клиенту (рисунок~\ref{fig:gateway-seq}).

\begin{figure}[H]
    \centering
    \includemermaid{03-gateway-seq}
    \caption{Диаграмма последовательности проксирования WebSocket}
    \label{fig:gateway-seq}
\end{figure}

\subsection{Декларативная маршрутизация}
Конфигурация маршрутов вынесена в YAML-файл, что позволяет добавлять новые микросервисы без пересборки шлюза:

\begin{code}{route-registry}{yaml}{Фрагмент route\_registry.yaml}
routes:
  - path: "/ws/route"
    service_url: "ws://router:8002/ws"
    type: "websocket"
  - path: "/tiles/{z}/{x}/{y}.pbf"
    service_url: "http://data_processor:8001/tiles"
    type: "http"
\end{code}

\section{Реализация клиентского приложения}

Интерфейс приложения разработан с использованием фреймворка Qt (модуль Qt Widgets) для реализации панелей управления и компонента \texttt{QWebEngineView} для отображения интерактивной карты. Архитектура UI построена по принципу разделения ответственности: логика взаимодействия с картой вынесена в JavaScript-модули, а управление состоянием приложения и сетевое взаимодействие — в Python-код.

\subsection{Пользовательский интерфейс и управление}

Главное окно приложения (рисунок~\ref{fig:client-main}) разделено на две функциональные области: интерактивную карту и боковую панель управления (Sidebar). Боковая панель содержит набор виджетов для управления процессом маршрутизации:

\begin{itemize}
    \item \textbf{Панель маршрутных точек (Points Panel):} Позволяет пользователю задавать начальную (Start), конечную (End) и промежуточные (Via) точки маршрута. Реализован функционал добавления, удаления и изменения порядка следования точек. Координаты могут быть введены вручную или выбраны кликом по карте.
    \item \textbf{Панель результатов (Routes Panel):} Отображает список найденных альтернативных маршрутов (K-Shortest Paths). Для каждого маршрута выводятся ключевые метрики: общая протяженность (км), расчетное время в пути (мин) и количество ребер графа. Выбор элемента в списке подсвечивает соответствующую траекторию на карте.
    \item \textbf{Управление симуляцией (Simulation Panel):} Блок элементов интерфейса («Sim Speed», «FPS», управление агентом), зарезервированный для подсистемы мультиагентного моделирования. В текущей версии системы эти элементы демонстрируют готовность архитектуры к внедрению динамических агентов (автомобилей), движущихся по построенным маршрутам, но их функционал пока ограничен базовой отладкой.
\end{itemize}

\begin{figure}[H]
    \centering
    \includegraphics[width=0.95\linewidth]{gui-routes}
    \caption{Главное окно приложения. Слева — панель управления с параметрами маршрута (K=5) и списком найденных путей. На карте фиолетовым цветом отображены 5 альтернативных маршрутов, проходящих через заданные точки.}
    \label{fig:client-main}
\end{figure}

\subsection{Система уровней детализации (LOD)}

Для обеспечения высокой производительности при отображении графа, содержащего более 250 000 ребер, реализована серверная фильтрация данных (Level of Detail). 

В отличие от классического подхода, где фильтрация происходит на клиенте (скрытие слоев через CSS), система реализует \textit{динамическую генерацию тайлов}. При запросе тайла сервер Data Processor проверяет текущий уровень масштабирования (Zoom Level) и модифицирует SQL-запрос к базе данных, исключая геометрию второстепенных дорог, невидимых на данном масштабе. Это позволяет радикально снизить объем передаваемых данных (размер MVT-тайла) и нагрузку на клиентский рендеринг.

Эффект от применения LOD продемонстрирован на рисунке~\ref{fig:client-lod}:
\begin{itemize}
    \item \textbf{(а) z9:} Отображаются только магистрали федерального значения (Motorway, Trunk).
    \item \textbf{(б) z11:} Добавляются дороги регионального значения (Primary, Secondary).
    \item \textbf{(в) z13:} Становится видна основная уличная сеть.
    \item \textbf{(г) z15:} Загружается полная геометрия, включая дворовые проезды и служебные дороги.
\end{itemize}

\begin{figure}[H]
    \centering
    \begin{tabular}{cc}
        \includegraphics[width=0.45\linewidth]{gui-lod-1} & \includegraphics[width=0.45\linewidth]{gui-lod-2} \\
        \small (а) Низкий масштаб (z9) & \small (б) Средний масштаб (z11) \\
        \includegraphics[width=0.45\linewidth]{gui-lod-3} & \includegraphics[width=0.45\linewidth]{gui-lod-4} \\
        \small (в) Высокий масштаб (z13) & \small (г) Максимальная детализация (z15)
    \end{tabular}
    \caption{Демонстрация работы системы LOD: прогрессивная детализация дорожной сети}
    \label{fig:client-lod}
\end{figure}

\subsection{Рендеринг векторных тайлов (MVT)}
Для корректного отображения сложной городской застройки (мосты над дорогами, тоннели) на стороне клиента реализована сортировка объектов по Z-индексу (Z-Order). Векторные тайлы, генерируемые сервером, содержат атрибут \texttt{layer}, который используется стилями MapLibre GL JS для определения порядка отрисовки. Без этого механизма дороги могли бы визуально перекрывать мосты, проходящие под ними.


\chapter{Технологический раздел}
\label{ch:testing}

\section{Методология экспериментального исследования}
Для оценки эффективности разработанной системы маршрутизации был выбран подход, основанный на использовании реальных геопространственных данных Московской агломерации (более 240 000 ребер). Тестовая выборка формируется путем генерации 1000 пар случайных точек внутри границ города, что обеспечивает репрезентативное покрытие различных сценариев: от коротких локальных поездок до трансгородских маршрутов.

Исследование проводилось в двух режимах:
\begin{itemize}
    \item \textbf{Latency Test (Латентность):} Последовательное выполнение запросов для измерения <<чистого>> времени отклика.
    \item \textbf{Throughput Test (Пропускная способность):} Параллельное выполнение запросов 16 конкурентными потоками.
\end{itemize}

\section{Анализ производительности}

\subsection{Временные характеристики и Cold Start}
Медианное время отклика системы составляет 6.2 секунды. Однако наблюдается значительный разброс значений (от 2.5 с до 40 с), что обусловлено эффектом <<холодного старта>> (Cold Start).
Профилирование показало, что система является I/O-bound (ограничена скоростью диска). Более 90\% времени выполнения запроса затрачивается на физическое чтение страниц данных PostgreSQL с диска (NVMe SSD) в Shared Buffers.
При повторном обращении к <<прогретому>> участку графа время отклика сокращается до 2.5 секунд.

Линейная зависимость времени выполнения от числа итераций алгоритма представлена на графике регрессии (рисунок~\ref{fig:latency-regression}). Выбросы в верхней части графика соответствуют <<холодным>> запросам, требующим полной загрузки данных с диска.

\begin{figure}[H]
    \centering
    \includegraphics[width=0.8\linewidth]{latency_regression} % Placeholder: throughput_complexity_3d used temporarily
    \caption{Регрессионный анализ времени отклика (с учетом Cold Start)}
    \label{fig:latency-regression}
\end{figure}

Зависимость времени поиска от сложности задачи ($N$ точек, $K$ альтернатив) представлена на 3D-графике (рисунок~\ref{fig:complexity-3d}).

\begin{figure}[H]
    \centering
    \includegraphics[width=0.8\linewidth]{throughput_complexity_3d}
    \caption{Зависимость времени поиска от сложности запроса}
    \label{fig:complexity-3d}
\end{figure}

Регрессионный анализ показывает линейную зависимость времени выполнения от числа итераций алгоритма, где каждая дополнительная итерация поиска добавляет в среднем 1.23 секунды к базовой задержке.

\subsection{Парадокс масштабируемости и MVCC}
При увеличении нагрузки в 16 раз (16 параллельных клиентов) среднее время отклика увеличилось всего на 9.7\% (с 6.2 до 6.8 с). Это контринтуитивное поведение объясняется архитектурой PostgreSQL и асинхронной моделью приложения.

\begin{figure}[H]
    \centering
    \includemermaid{04-mvcc-seq}
    \caption{Взаимодействие клиентов и БД в режиме конкурентности (MVCC)}
    \label{fig:mvcc-seq}
\end{figure}

Механизм MVCC (Multiversion Concurrency Control) (рисунок~\ref{fig:mvcc-seq}) позволяет читающим транзакциям не блокировать друг друга. Пока один процесс ожидает завершения длительной операции ввода-вывода (I/O Wait), планировщик ОС переключает процессорное время на обработку других запросов. В результате, несмотря на высокую латентность отдельных запросов, общая пропускная способность системы утилизирует ресурсы CPU максимально эффективно (87\% загрузки против 12\% в однопоточном режиме).

\section{Оценка качества маршрутизации}

\subsection{Метрика извилистости (Tortuosity)}
Для оценки оптимальности найденных маршрутов использовался коэффициент извилистости $T = L_{route} / L_{euclid}$.
Медианное значение $T = 1.36$ указывает на то, что типичный маршрут всего на 36\% длиннее прямой линии, что является хорошим показателем для плотной городской застройки. Отсутствие значений $T > 3.0$ подтверждает отсутствие критических петель и неоправданных объездов.

\subsection{Надежность (Yield Rate)}
Тепловая карта успешности поиска (рисунок~\ref{fig:yield-heatmap}) демонстрирует 100\% надежность для базовых запросов ($K=1$).

\begin{figure}[H]
    \centering
    \includegraphics[width=0.8\linewidth]{throughput_yield_heatmap}
    \caption{Тепловая карта успешности построения маршрутов}
    \label{fig:yield-heatmap}
\end{figure}

Снижение успешности при $K > 3$ (до 41\% для $N=5, K=5$, правый верхний угол тепловой карты) обусловлено топологическими ограничениями дорожной сети, где физически невозможно построить 5 независимых маршрутов между заданными точками без существенного перекрытия. Основными причинами сбоев остаются попадание точек в изолированные анклавы графа (NoRouteFound, 14.8\%).

\section{Сравнительная характеристика}

Разработанная система уступает решениям класса In-Memory (OSM Routing Machine) по скорости отклика, однако выигрывает в гибкости и оперативной памяти. Сводная таблица сравнения приведена ниже (таблица~\ref{tab:final-comparison}).
Потребление RAM при сборке графа составляет всего 370 МБ (против гигабайтов у аналогов), а любые изменения в дорожной сети (перекрытия, ремонт) учитываются мгновенно через изменение весовых коэффициентов в SQL-запросах, без необходимости многочасовой перестройки индексов.

\begin{table}[H]
    \centering
    \caption{Сравнение производительности: Разработанная система vs OSRM}
    \label{tab:final-comparison}
    \begin{tabular}{|l|c|c|c|}
        \hline
        \textbf{Характеристика} & \textbf{OSRM (In-Memory)} & \textbf{Разработанная система} & \textbf{Вывод} \\ \hline
        Время отклика & 10 мс & 6000 мс & OSRM быстрее (600x) \\ \hline
        Обновление графа & 30--60 мин & 0 мс (Instant) & Наша система гибче \\ \hline
        Потребление RAM & 4--8 ГБ & 370 МБ & Наша система экономнее \\ \hline
        Динамические веса & Нет & Да & Критическое преимущество \\ \hline
    \end{tabular}
\end{table}

Данная архитектура оптимальна для логистического планирования и диспетчерских задач, где возможность учета сложных динамических ограничений важнее мгновенного отклика.

\chapter*{ЗАКЛЮЧЕНИЕ}
\addcontentsline{toc}{chapter}{ЗАКЛЮЧЕНИЕ}

В ходе выполнения курсовой работы была спроектирована и реализована распределенная система построения маршрутов на основе микросервисной архитектуры.

В результате исследования были получены следующие основные результаты:
\begin{enumerate}
    \item Обоснован выбор технологического стека \textbf{PostgreSQL + PostGIS + pgRouting}, обеспечивающего баланс между производительностью и гибкостью динамической маршрутизации.
    \item Разработан алгоритм \textbf{Grid Partitioning} (плиточная нарезка), позволивший снизить потребление оперативной памяти при построении графа Москвы с 12 ГБ до 370 МБ и сократить время сборки до 13 минут.
    \item Реализован механизм загрузки данных из OpenStreetMap, устойчивый к таймаутам и лимитам публичных API, обеспечивающий автоматическую актуализацию дорожной сети.
    \item Создан клиентский интерфейс на базе \textbf{PyQt5} и \textbf{MapLibre GL JS}, поддерживающий плавную визуализацию векторных тайлов и интерактивное взаимодействие с картой.
    \item Проведено нагрузочное тестирование, показавшее, что система способна обрабатывать запросы маршрутизации со средним временем отклика 6.2 секунды. Подтверждена эффективность параллельной обработки запросов.
\end{enumerate}

Полученные характеристики системы (время отклика 6 секунд) ограничивают ее применение в качестве навигатора реального времени для конечного пользователя, однако полностью удовлетворяют требованиям к системам логистического планирования и диспетчеризации, где ключевым фактором является возможность учета сложных динамических ограничений, невозможных в статических графовых движках.


% ВСТАВКА БИБЛИОГРАФИИ (отключить для скорости)
\makebibliography

\end{document}